\documentclass[letter]{book}

\usepackage{hyperref}

%% symbol definitions
\RequirePackage{ss-common}

%% book formatting commands
\pagestyle{headings}
\usepackage{lettrine}
\usepackage[T1]{fontenc}
\usepackage{textcomp}
\usepackage[urw-garamond]{mathdesign}
% command to move to the next left-facing page
\makeatletter
\def\advancetoleft{\if@openright\clearpage\else\cleardoublepage\fi}
\makeatother

\title{Computations with spectral sequences}
\author{Eric Peterson}
\date{}

\begin{document}

\maketitle
\tableofcontents
\clearpage

\section{Foreword}

Mathematics is a field where computations lead theory, and this is especially evident in the subfield of algebraic topology, which is positively rife with computations.  These often take the form of spectral sequences, which are notorious among students of any field that makes use of homological algebra for being pathologically cryptic and complex.  Nevertheless, their utility is immense, and students, often with much groaning, at least learn to stomach the sight of them, if not fully embrace the idea of computing with one.

There are many reasons spectral sequences are viewed as impossibly complex, large parts of which are due to the following two reasons.  First, spectral sequences are often triply-indexed --- and each index is often infinite, or bi-infinite, or indexed over a group more complicated than the integers!  This means that an enormous amount of information is available in a spectral sequence, which begets the second point: effective computation with a spectral sequence appears to require that one keep an outlandish number of things in mind while working, along with an array of subtle tricks and facts from elsewhere in topology, not presently visible on the page.  In turn, these have lead to a derth of textbooks covering the art of computing with spectral sequences; if they're so difficult to think about, then the situation is even worse when trying to linearize them into writing and then typeset the whole mess.  For this reason, knowing how to compute with spectral sequences is often referred to as an ``oral tradition,'' passed down in ritual form from advisor to student, behind closed doors and with endless scratch paper.

The purpose of this text is to fill this gap.  In conversation with an expert, time plays the role of linearizer, as one watches the spectral sequence play out on a page in real time.  Our goal is to turn these conversations into text, where the linearization instead takes place across pages, in the form of an elementary school student's ``flip book.''  On each page the reader can find a single step of the larger computation highlighted and dissected, then turn to the next to find the diagram slightly modified, as in real-time.  This should dramatically ease the learning curve for students who are interested in spectral sequences but who don't enjoy ready access to lunches with Doug Ravenel and crew.

This book does not have exercises; instead, it is written more like a solutions manual for a text that does not exist.  However, the methods described are extremely general, and the reader looking to try them out for himself should be able to pick a favorite space and plug it into these machines, following roughly the same process to compute its associated invariants.  For this reason, the examples worked here have been selected with illustration kept in mind rather than exhaustiveness.

An important thing to remind the reader of is that spectral sequences, as massive mathematical machines, are designed to take their users' minds off the details of a problem.  Some of these details will be addressed and discussed lightly in the text surrounding the computations, but the uninterested, bored, or befuddled reader should not hesitate to skip over these parts of the text for now.  In the same way, schoolchildren are taught arithmetic algorithms long before they investigate what makes the algorithms tick, and in this intervening period the utility of knowing how to perform long division is not diminished.

We should also immediately mention other textbooks on this subject.  McCleary's book \textit{A User's Guide to Spectral Sequences} is excellent and contains all of the details we omit here and then some.  Mosher \& Tangora's \textit{Cohomology Operations and Applications in Homotopy Theory} centers around the interactions of the Steenrod algebra with spectral sequences, and is rife with the computations that spurred the development of this field.  Every homological algebra textbook in existence (Weibel's \textit{Homological algebra}, Cartan and Eilenberg's \textit{Homological algebra}, \ldots) contains a section on the construction and maintenance of spectral sequences, where technical details can be found.  Hatcher has made available an unfinished book project on spectral sequences at \url{http://www.math.cornell.edu/~hatcher/SSAT/SSATpage.html}.  Miller has published course notes that use in a central way the EHP spectral sequence, available in full at \url{http://www-math.mit.edu/~hrm/papers/} and in the process of being converted to \LaTeX.  Ravenel's \textit{Complex cobordism and stable homotopy groups of spheres} remains the standard reference for the analysis of the beginning of the Adams spectral sequence for the sphere.  And, of course, there are many others.

Finally, this is a draft version of this textbook, compiled on \today.  I'm sure that it's rife with errors, inconsistencies, omissions, and generally confused language, and I would greatly appreciate any or all of corrections, remarks, and expansions.  I can easily be reached at \texttt{ericp@math.berkeley.edu}.  This project progresses slowly, as I tend to work on it only when I'm stuck on and tired of my other mathematical projects, but I hope that it grows into something genuinely useful as it goes.

Drafts of this document are available at \url{http://math.berkeley.edu/\~ericp/ss-book/main.pdf}, and the software used to generate it is available in the directory \url{http://math.berkeley.edu/\~ericp/ss-book/}.

\section{Acknowledgements}

People who directly taught me: Matthew Ando

People who have helped substantially with this book, through contributions or editing: Aaron Mazel-Gee % Akhil Mathews

Topologists whose computations have had a profound influence on me: Mike Hill, Mike Hopkins, Robert Mosher, Justin Noel, Doug Ravenel, Neil Strickland, Martin Tangora, W.\ Steve Wilson

More facilitators: Peter Teichner, Constantin Teleman

Locations and funding sources: UC-Berkeley, MPIM-Bonn

\chapter{Spectral sequences in general}

\lettrine[lines=1]{B}{efore} we start in on computations with spectral sequences, we should take a moment to outline what they are and where they come from.  Once we've pinned these down, we will also mention some of the most common complications and useful structures. 

\section{Homology theories}

Spectral sequences arise naturally in homological algebra, which is the study in the abstract of where homology functors come from.  Since this book is geared toward algebraic topologists, we will not be so abstract; instead, a (reduced) homology functor for us is a sequence of functors $(\tilde H_n)_{n \in \Z}: Ho(\CatOf{PointedSpaces}) \to \CatOf{AbelianGroups}$ from the category of pointed homotopy types to abelian groups which collectively satisfy the following two axioms:
\begin{itemize}
\item Wedge sum: For any collection of spaces\footnote{Throughout this book, we will suppress the basepoint we carry along with our spaces.  It's an important technicality, but it's not worth dwelling on constantly by bringing into the notation.} $\{X_\alpha\}_{\alpha \in A}$, we have a natural isomorphism \[\tilde H_n \left( \bigvee_\alpha X_\alpha \right) \cong \bigoplus_\alpha \tilde H_n X_\alpha.\]
\item Triangulation: For $A$ a subspace\footnote{We require $i: A \to X$ to be quite reasonable, namely a cofibration.  For example, the inclusion of a subcomplex counts.} of $X$, the ``short exact sequence'' \[A \xrightarrow{i} X \xrightarrow{p} X/A\] of spaces begets a long exact sequence
\begin{center}
\begin{tikzpicture}
    \matrix (m) [matrix of math nodes,
        row sep=2.2em, column sep=1.5em,
        text height=1.5ex, text depth=0.25ex]{
       &              & \cdots       & \tilde H_{n+1}(X/A) & \\
       & \tilde H_n A & \tilde H_n X & \tilde H_n(X/A) & \\
       & H_{n-1} A    & \cdots. \\
    };
    \path[normal line]
        (m-1-3) edge (m-1-4)
        (m-1-4) edge[out=0,in=180] (m-2-2)
        (m-2-2) edge (m-2-3)
        (m-2-3) edge (m-2-4)
        (m-2-4) edge[out=0,in=180] (m-3-2)
        (m-3-2) edge (m-3-3);
\end{tikzpicture}
\end{center}
The middle maps are specified by functoriality, but the maps labeled $\partial$ are new data.
\end{itemize}
\TODO: Mention unreduced.

These axioms alone can be used to compute a small handful of things.  For instance, the first axiom tells us that the homology of a point must vanish, since $\pt \vee \pt \simeq \pt$.  To see the utility of the second axiom, let $X$ be a $(d+1)$-dimensional hemisphere, and let $A$ be the inclusion of the equatorial band, itself a $d$-dimensional sphere.  The space $X$ is homotopy equivalent to a point, so has vanishing homology, whereas the quotient $X / A$ is homeomorphic to a $(d+1)$-dimensional sphere.  The long exact sequence in homology reads \[\cdots \to \tilde H_{n+1} S^{d+1} \xrightarrow{\partial} \tilde H_n S^d \xrightarrow{\tilde H_n i} \tilde H_n \pt \xrightarrow{\tilde H_n p} \tilde H_n S^{d+1} \xrightarrow{\partial} \tilde H_{n-1} S^d \to \cdots.\]  Hence, the homology of the $(d+1)$-sphere is exactly the homology of the $d$-sphere, shifted up by one degree.

\TODO: Mention cohomology.
\TODO: A useful fact is $H_* \colim F = \colim H_* F$.

\section{Filtrations and spectral sequences}

This is all well and good, and one can compute a great many things manually by specifying $H_* S^0$ and working with these two axioms from there.  For more complex situations, manual computations become tedious, and this is where spectral sequences enter the picture.  To perform the homology computation of a complex space $X$, we must first break it down into a sequence of simple spaces $X_q$, each including into the next.  Not only should all of them include into $X$, but we should have $X = \colim_q X_q$.  On the other end, we require $X_{-1} = \pt$.  Here is a diagram of the situation:
\begin{center}
\begin{tikzpicture}
    \matrix (m) [matrix of math nodes,
         row sep=3em, column sep=3em,
         text height=1.5ex,
         text depth=0.25ex]{
             \cdots & X_{q-1} & X_q & X_{q+1} & \cdots & X. \\
    };
    \path[normal line]
	    (m-1-1) edge[into] (m-1-2)
		(m-1-2) edge[into] node[above]{$i_{q-1}$} (m-1-3)
		(m-1-3) edge[into] node[above]{$i_q$} (m-1-4)
		(m-1-4) edge[into] (m-1-5)
		(m-1-5) edge[into] (m-1-6);
\end{tikzpicture}
\end{center}
We're seeking to relate the homology of $X$ to the homologies of these pieces $X_q$.  Looking back at our axioms for a homology theory, we do see that inclusions play a special role in the triangulation axiom, but to apply the triangulation axiom we must also consider various quotients.  We extend our diagram to match:
\begin{center}
\begin{tikzpicture}
    \matrix (m) [matrix of math nodes,
         row sep=3em, column sep=3em,
         text height=1.5ex,
         text depth=0.25ex]{
             \cdots & X_{q-1} & X_q & X_{q+1} & \cdots & X. \\
			        & F_{q-1} & F_q & F_{q+1} \\
    };
    \path[normal line]
	    (m-1-1) edge[into] (m-1-2)
		(m-1-2) edge[into] node[above]{$i_{q-1}$} (m-1-3)
		        edge[onto] node[right]{$p_{q-1}$} (m-2-2)
		(m-1-3) edge[into] node[above]{$i_q$} (m-1-4)
		        edge[onto] node[right]{$p_q$} (m-2-3)
		(m-1-4) edge[into] (m-1-5)
		        edge[onto] node[right]{$p_{q+1}$} (m-2-4)
		(m-1-5) edge[into] (m-1-6);
\end{tikzpicture}
\end{center}
Now we apply homology $\tilde H^*$ to our diagram, and in doing so we also apply the triangulation axiom to each of these angled arms:
\begin{center}
\begin{tikzpicture}
    \matrix (m) [matrix of math nodes,
         row sep=3em, column sep=3em,
         text height=1.5ex,
         text depth=0.25ex]{
             \cdots & \tilde H_* X_{q-1} & \tilde H_* X_q & \tilde H_* X_{q+1} & \cdots & \tilde H_* X. \\
			        & \tilde H_* F_{q-1} & \tilde H_* F_q & \tilde H_* F_{q+1} \\
    };
    \path[normal line]
	    (m-1-1) edge (m-1-2)
		(m-1-2) edge node[above]{$\tilde H_* i_{q-1}$} (m-1-3)
		        edge node[right]{$\tilde H_* p_{q-1}$} (m-2-2)
		(m-1-3) edge node[above]{$\tilde H_* i_q$} (m-1-4)
		        edge node[right]{$\tilde H_* p_q$} (m-2-3)
		(m-1-4) edge (m-1-5)
		        edge node[right]{$\tilde H_* p_{q+1}$} (m-2-4)
		(m-1-5) edge (m-1-6)
		(m-2-3) edge node[above]{$\partial$} (m-1-2)
		(m-2-4) edge node[above]{$\partial$} (m-1-3);
\end{tikzpicture}
\end{center}
Note that each of these maps $\partial$ is \emph{not} degree-preserving\footnote{A key to successfully doing homological algebra successfully is to suppress as many indices as possible, so we don't draw this in the diagram.} but shifts the degree down by $1$.

Now that we have this picture, we are tasked with tying this discussion up and saying something meaningful about $\tilde H_* X$.  Before formalizing the process, we will describe its goal.  Suppose that we pick some homology class $\alpha \in \tilde H_* F_q$; the question we then pose is whether $\alpha$ is in some way visible in $H_* X$.  The only map we have in front of us by which we can push back up into the $X$es is $\partial$, so we produce an element $\partial \alpha \in H_{*-1} X_{q-1}$.  Na\"ively, we'd want to then push forward into $H_* X$ by tracking the maps to the right, but because the triangles in our diagram are exact we immediately know that $(\tilde H_* i_q) \circ \partial \alpha = 0$.  We must be more creative.

Another thing we could try to do is to find an element $\beta \in \tilde H_* X_q$ for which $(\tilde H_* p_q) \beta = \alpha$.  Again employing exactness of the triangle, such a $\beta$ exists exactly when $\partial \alpha = 0$.  However, at the moment we have no way of telling whether this is the case, since the rules of the game are that we only understand the groups $\tilde H_* F_*$.  So, to get back into the land of things we understand, we follow the vertical map down to produce $(\tilde H_* p_q) \circ \partial \alpha$.

At this point there are two options.  First, $(\tilde H_* p_q) \circ \partial \alpha$ could be nonzero, in which case $\partial \alpha$ itself must have been nonzero, and there is no hope for producing $\beta$.  In this case, we should discard $\alpha$ as an unfortunate artifact of the filtering process, without contribution to the total homology.  On the other hand, if $(\tilde H_* p_q) \circ \partial \alpha = 0$, it's possible that either $\partial \alpha = 0$ or merely that $\partial \alpha \in \ker \tilde H_* p_q$.  But, in either case, we can employ the exactness of the next triangle in the sequence to preimage the element $\partial \alpha$ through the map $\tilde H_* i_{q-1}$ to produce an element $(\tilde H_* i_{q-1})^{-1} \partial \alpha$, with which we can play the same game.

Eventually, however, we will hit the bottom of our filtration.  If we can play this game all the way back to then, then we have produced an element $\gamma = (\tilde H_* i_*)^{\circ (-q)} \partial \alpha$ for which $(\tilde H_* i_*)^{\circ q} \gamma = \alpha$.  However, because $X_{-1} = \pt$, we know that $\gamma = 0$, and hence $\partial \alpha = (\tilde H_* i_*)^{\circ q}(0) = 0$, and we win --- $\beta$ exists!

This process is formalized by packaging up these composites.  We write $E^1_{*, q} = \tilde H_* F_q$, and the map $(\tilde H_* p_*) \circ \partial$ is called $d^1: E^1_{*, q} \to E^1_{*-1, q-1}$.  One quickly checks that $d^1$ is a differential, as the two maps in the middle of $d^1 \circ d^1 = (\tilde H_* p_*) \circ \partial \circ (\tilde H_* p_*) \circ \partial$ belong to the same exact triangle, and hence compose to zero.  We are interested only in keeping classes in the kernel of the outgoing $d^1$ while deleting all the classes in the kernel of the incoming $d^1$, and so advancing to the next stage in the game corresponds exactly to taking cohomology against the differentials $d^1$.  This cohomology group we label $E^2_{*, q}$.  By a small miracle, it turns out that this same quotient is what is required to eliminate the indeterminacy in picking the preimage $(\tilde H_* p_{q-1}) \circ (\tilde H_* i_{q-1})^{-1} \circ \partial \alpha$, and this composite we label $d^2: E^2_{*, q} \to E^2_{*-1, q-2}$.  This pattern in producing differentials and computing their cohomology continues, and in general we have groups $E^r_{*, q}$, which are sub-quotients of $E^{r-1}_{*, q}$, and differentials $d^r: E^r_{*, q} \to E^r_{*-1, q-r}$.  The index $r$ is called the ``page'' or ``sheet,'' and altogether this data forms a ``spectral sequence.''

\section{Convergence and the endgame}

One can produce spectral sequences for cohomology as well, using an identical setup.  The only difference is in the endgame: in homology, we kept lowering filtration degree, so we eventually hit the bottom and deduced something about our element $\alpha$.  In cohomology, we will instead \emph{raise} filtration degree, and so we will never hit bottom and be able to conclude something solid.  We will, however, continuously march toward $\tilde H^* X$ with which filtration degree we climb up, and so our spectral sequence will compute something about $\lim_q H^* X_q$, the limit of the cohomology groups.  Whether this compares well with $H^* X$ is one of the things we discuss now.

The general theory of spectral sequences is quite wild, and it is possible to construct spectral sequences not arising naturally from a filtration in the way we've described.  However, almost all of the examples witnessed in the wild (and certainly those with which one should learn to compute) do come from this construction, and assuming we're in this situation simplifies the theory of convergence considerably.

Label the groups $H_* X_q$ of the above construction by $F_{*, q}$, and label $H_* X$ by $G_*$.  The spectral sequence is said to be \ldots
\begin{itemize}
\item \ldots \defn{weakly convergent} (to $G_*$) if $\colim_q F_{*, q} = G_*$ and $E^\infty_q = F_{*, q} / F_{*, q-1}$.
\item \ldots \defn{convergent} if it is weakly convergent and furthermore $\lim_q F_{*, q} = 0$.
\item \ldots \defn{strongly convergent} if it is convergent and furthermore $\lim^1_q F_{*, q} = 0$.
\item \ldots \defn{conditionally convergent} if $\lim F_{*, q} = 0$.\footnote{In this situation, the filtration is said to be \defn{Hausdorff}.}
\end{itemize}

The first three conditions neatly summarize what extra steps we will need to take in the end to compare the ``result'' of our spectral sequence with the target of convergence.  In the case of strong convergence, we need only to deal with extension problems.  The individual homology groups $G_p$ are, by construction, sliced up and scattered through the homology groups $\{E^\infty_{p, q}\}_q$ as $q$ ranges.  To recover $G_p$ from this sequence, we are faced with a nest of extension problems: there is some intermediate group extending $E^\infty_{p, 0}$ by $E^\infty_{p, 1}$, which in turn has an intermediate group extending it by $E^\infty_{p, 2}$, and so forth.  In the strongly convergent case, the limit of this process yields $G_p$.

In the convergent case, we are faced exactly with the issue presented by the cohomological spectral sequence above.  We can attempt to solve the extension problem, just as before, but the resulting groups $G'_p$ sit in a short exact sequence $0 \to G'_p \to G_p \to \lim^1_q F_{*, q} \to 0$ obstructing honest equality, so we must also address this.

In the weakly convergent case, we are faced with the above two issues, but we additionally ... uh actually I'm not sure what we have to do.  Throw in the intersection as a summand?

The conditionally convergent case is differently flavored from the rest.  Conditional convergence on its own is worse than weak convergence, but it appears frequently, and there are various extra mild assumptions, easily verified in practice, that turn conditional convergence into strong convergence.  For example, if $F_{*, q}$ stabilizes for $q \ll 0$, the spectral sequence converges conditionally, and if $\lim^1_q E^\infty_{*, q} = 0$, the convergence is strong.

There are two more vocabulary words worth knowing: in the case $X_{-1} = \pt$, the spectral sequence lives entirely on one half of the full doubly-integer-indexed plane, and so is called a half-plane spectral sequence.  In the homological case, where the differentials eventually land in the unoccupied half-plane, the associated spectral sequence is said to have \defn{exiting differentials}.  In the cohomological case, where all differentials eventually land in the occupied half-plane, the associated spectral sequence is said to have \defn{entering differentials}.

In the exiting case, we have few convergence issues to worry about: provided the filtration is Hausdorff, we have strong convergence.  If the differentials are entering, however, we need conditional convergence together with the vanishing $\lim^1$-term to get strong convergence.


\section{Grading conventions and multiplicative structures}

Pairings.  The Leibniz rule.

\chapter{Atiyah-Hirzebruch-Serre}

\section{The Atiyah-Hirzebruch spectral sequence}

The essential building blocks of the spaces in non-pathological topology (including algebraic topology) are the unit balls $D^n$ of dimension $n$, and their surface spheres $S^{d-1}$ of dimension $(d-1)$.  A topological space $X$ is said to be a CW-complex when it can be decomposed into a sequence of spaces $X^{(d)}$, called $d$-skeleta\footnote{Skeleta is the mathematician's plural of skeleton.}, such that $X^{(-1)} = \pt$ a single point and $X^{(d+1)}$ is formed from $X^{(d)}$ by gluing in (unpointed) $(d+1)$-balls along their $d$-spherical surface shells, and such that $X$ is given as the colimit of the $X^{(d)}$ as $d$ grows large.  These are somehow the most reasonable spaces on which we can ``do homotopy theory,'' and from here on out all our spaces will be assumed to be CW-complexes\footnote{The exact decomposition into the spaces $X^{(d)}$ isn't so important, just that there exists one.}.

This gives an ascending filtration of $X$ by $X^{(d)}$ which is Hausdorff (the condition on $X^{(-1)}$) and exhaustive (the condition $X = \colim_d X^{(d)}$).  Moreover, the filtration quotients are easy to compute: the cofiber of the map $X^{(d-1)} \into X^{(d)}$ collapses the $(d-1)$-skeleton to a point, to which all the $d$-cells get attached, resulting in a bouquet of spheres $X^{(d)} / X^{(d-1)} \simeq \bigvee_\alpha S^d_\alpha$ in filtration grading $d$.  Selecting our favorite homology theory $h_*$ and cohomology theory $h^*$, this gives a pair of spectral sequences with signatures
\begin{align*}
E^1_{s, t} = h_s \left(\bigvee_\alpha S^t_\alpha\right) = h_{s-t}(\pt) & \convergeto h_s X, & d^r: E^1_{s, t} & \to E^1_{s-1, t-r}, \\
E_1^{s, t} = h^s \left( \bigvee_\alpha S^t_\alpha \right) = h^{s-t}(\pt) & \convergeto h^s X, & d_r: E_1^{s, t} & \to E_1^{s+1, t+r}.
\end{align*}
In fact, we can do better: the differential on the first pages of these spectral sequences is exactly the differential that appears in the $(s-t)$th degree of the cellular chain complex for computing cohomology with $h^{s-t}(\pt)$-coefficients.  This spectral sequence also carries the structure of a $h^*(\pt)$-module in the case that $h$ takes its values in rings, though not with this grading.  \TODO: Straighten out this grading discussion.  Putting all this together produces the more familiar form of these spectral sequences:
\begin{align*}
E^2_{p, q} = H^{cell}_p(X; h_q(\pt)) & \convergeto H_{p+q} X, & d^r_{p, q}: E^r_{p, q} & \to E^r_{p-r, q+r-1}, \\
E_2^{p, q} = H_{cell}^p(X; h^q(\pt)) & \convergeto H^{p+q} X, & d_r^{p, q}: E_r^{p, q} & \to E_r^{p+r, q-r+1}.
\end{align*}

\section{$H^* \CP^\infty$}

The motivic cell decomposition.

\section{$H^* \RP^\infty$}

The motivic cell decomposition.

\section{$KU^* B\Z/2$}

Even-concentrated, but has extension problems.  See Strickland's bestiary.  This might be hard to do before the Gysin sequence description of $h^* \RP^\infty$...

\section{The Serre spectral sequence}

The $E_1$-page is easy, but $d_1$ is hard.  Multiplicative structure.

\advancetoleft
\section{$H^* \CP^\infty$ redux}
\vspace*{\fill}
Consider the spherical fibration \[S^1 \to \mathbb{C}^\infty \setminus \{0\} \to \mathbb{C}\mathrm{P}^\infty.\]  The total space, $\mathbb{C}^\infty \setminus \{0\} \simeq S^\infty$, is contractible, hence has vanishing cohomology.  The fiber $S^1$ has known cohomology groups, $H^*(S^1; \mathbb{Z}) = \Lambda[e]$.  We know that, $\mathbb{C}\mathbb{P}^\infty$ is connected, and hence we can compute, $H^0(\mathbb{C}\mathbb{P}^\infty; H^* S^1)$ --- it has two free, generators $1$ and $e$ in $q$-degrees $0$ and $1$.\vspace*{\fill}
\newpage
\vspace*{\fill}
\begin{tikzpicture}[group/.style={},auto]
\draw[use as bounding box,white] (-1.000000, -1.000000) rectangle (9.000000,6.000000);
\path [draw, gray!50, very thin] (0.000000, 0.000000) grid (8.000000, 5.000000);
\draw[->,gray!50,thin] (0, 0) to (9.000000, 0);
\draw[->,gray!50,thin] (0, 0) to (0, 6.000000);
\node[label,gray!50] at (0, -0.5) {$0$};
\node[label,gray!50] at (1, -0.5) {$1$};
\node[label,gray!50] at (2, -0.5) {$2$};
\node[label,gray!50] at (3, -0.5) {$3$};
\node[label,gray!50] at (4, -0.5) {$4$};
\node[label,gray!50] at (5, -0.5) {$5$};
\node[label,gray!50] at (6, -0.5) {$6$};
\node[label,gray!50] at (7, -0.5) {$7$};
\node[label,gray!50] at (8, -0.5) {$8$};
\node[label,gray!50] at (9.000000, -0.5) {$p$};
\node[label,gray!50] at (-0.5, 0) {$0$};
\node[label,gray!50] at (-0.5, 1) {$1$};
\node[label,gray!50] at (-0.5, 2) {$2$};
\node[label,gray!50] at (-0.5, 3) {$3$};
\node[label,gray!50] at (-0.5, 4) {$4$};
\node[label,gray!50] at (-0.5, 5) {$5$};
\node[label,gray!50] at (-0.5, 6.000000) {$q$};
\begin{scope}
\clip (-2.000000, 0.000000) rectangle (8.000000, 5.000000);
\end{scope}
\node[group] (one) at (0.000000, 0.000000) {$\mathbb{Z}$};
\node[group,color=red] (e) at (0.000000, 1.000000) {$\mathbb{Z}$};
\node[label,color=red,left=of e] {$e$};
\end{tikzpicture}
\vspace*{\fill}
\newpage
\vspace*{\fill}
The Serre spectral sequence associated to a singular
    theory is a first-quadrant spectral sequence, and hence $E_2^{p, q} = 0$
    whenever $p$ or $q$ is negative.  The differentials have the type signature
    \[d_r: E_r^{p, q} 	o E_r^{p+r, q-r+1}\] and hence if the class $e$ is to be
    killed by a differential --- and it must, since $H^*(S^\infty, \mathbb{Z})
    = \mathbb{Z}$ --- it must happen on this page.  Therefore, there must be a
class $x$ in $E_2^{2, 0} = H^2(\mathbb{C}\mathrm{P}^\infty; H^0(S^1;
\mathbb{Z}))$ with $d_2(e) = x$.\vspace*{\fill}
\newpage
\vspace*{\fill}
\begin{tikzpicture}[group/.style={},auto]
\draw[use as bounding box,white] (-1.000000, -1.000000) rectangle (9.000000,6.000000);
\path [draw, gray!50, very thin] (0.000000, 0.000000) grid (8.000000, 5.000000);
\draw[->,gray!50,thin] (0, 0) to (9.000000, 0);
\draw[->,gray!50,thin] (0, 0) to (0, 6.000000);
\node[label,gray!50] at (0, -0.5) {$0$};
\node[label,gray!50] at (1, -0.5) {$1$};
\node[label,gray!50] at (2, -0.5) {$2$};
\node[label,gray!50] at (3, -0.5) {$3$};
\node[label,gray!50] at (4, -0.5) {$4$};
\node[label,gray!50] at (5, -0.5) {$5$};
\node[label,gray!50] at (6, -0.5) {$6$};
\node[label,gray!50] at (7, -0.5) {$7$};
\node[label,gray!50] at (8, -0.5) {$8$};
\node[label,gray!50] at (9.000000, -0.5) {$p$};
\node[label,gray!50] at (-0.5, 0) {$0$};
\node[label,gray!50] at (-0.5, 1) {$1$};
\node[label,gray!50] at (-0.5, 2) {$2$};
\node[label,gray!50] at (-0.5, 3) {$3$};
\node[label,gray!50] at (-0.5, 4) {$4$};
\node[label,gray!50] at (-0.5, 5) {$5$};
\node[label,gray!50] at (-0.5, 6.000000) {$q$};
\node[group] (e) at (0.000000, 1.000000) {$\mathbb{Z}$};
\node[label,left=of e] {$e$};
\node[group] (one) at (0.000000, 0.000000) {$\mathbb{Z}$};
\begin{scope}
\clip (-2.000000, 0.000000) rectangle (8.000000, 5.000000);
\end{scope}
\node[group,color=red] (x) at (2.000000, 0.000000) {$\mathbb{Z}$};
\node[label,color=red,below=of x] {$x$};
\draw[->,color=red] (e) to (x);
\end{tikzpicture}
\vspace*{\fill}
\newpage
\vspace*{\fill}
But, if $E_2^{2, 0} =
        H^2(\mathbb{C}\mathrm{P}^\infty; H^0(S^1; \mathbb{Z}))$ is nonzero,
        then $E_2^{2, 1} = H^2(\mathbb{C}\mathrm{P}^\infty; H^1(S^1;
        \mathbb{Z}))$ is also nonzero, since $H^0(S^1; \mathbb{Z}) \cong
        H^1(S^1; \mathbb{Z}) \cong \mathbb{Z}$.  The Serre spectral sequence
        is multiplicative, and so we already have a name for this element: $e
        \cdot x$.  Moreover, $d_2$ is a derivation, so \begin{align*} d_2(e
        \cdot x) & = d_2(e) \cdot x + (-1) e \cdot d_2(x) \ & = x^2 + 0 =
        x^2. \end{align*}  For degree reasons, $e \cdot x$ must also be killed
        on the $E_2$-page, and hence $x^2$ must exist in $E_2^{4, 0}$.  This
        pattern continues, as $d_2(e \cdot x^n) = x^{n+1} + (-1) e \cdot n
        x^{n-1} \cdot 0 = x^{n+1}$.\vspace*{\fill}
\newpage
\vspace*{\fill}
\begin{tikzpicture}[group/.style={},auto]
\draw[use as bounding box,white] (-1.000000, -1.000000) rectangle (9.000000,6.000000);
\path [draw, gray!50, very thin] (0.000000, 0.000000) grid (8.000000, 5.000000);
\draw[->,gray!50,thin] (0, 0) to (9.000000, 0);
\draw[->,gray!50,thin] (0, 0) to (0, 6.000000);
\node[label,gray!50] at (0, -0.5) {$0$};
\node[label,gray!50] at (1, -0.5) {$1$};
\node[label,gray!50] at (2, -0.5) {$2$};
\node[label,gray!50] at (3, -0.5) {$3$};
\node[label,gray!50] at (4, -0.5) {$4$};
\node[label,gray!50] at (5, -0.5) {$5$};
\node[label,gray!50] at (6, -0.5) {$6$};
\node[label,gray!50] at (7, -0.5) {$7$};
\node[label,gray!50] at (8, -0.5) {$8$};
\node[label,gray!50] at (9.000000, -0.5) {$p$};
\node[label,gray!50] at (-0.5, 0) {$0$};
\node[label,gray!50] at (-0.5, 1) {$1$};
\node[label,gray!50] at (-0.5, 2) {$2$};
\node[label,gray!50] at (-0.5, 3) {$3$};
\node[label,gray!50] at (-0.5, 4) {$4$};
\node[label,gray!50] at (-0.5, 5) {$5$};
\node[label,gray!50] at (-0.5, 6.000000) {$q$};
\node[group] (x) at (2.000000, 0.000000) {$\mathbb{Z}$};
\node[label,below=of x] {$x$};
\node[group] (e) at (0.000000, 1.000000) {$\mathbb{Z}$};
\node[label,left=of e] {$e$};
\node[group] (one) at (0.000000, 0.000000) {$\mathbb{Z}$};
\begin{scope}
\clip (-2.000000, 0.000000) rectangle (8.000000, 5.000000);
\draw[->] (e) to (x);
\end{scope}
\node[group,color=red] (x2) at (4.000000, 0.000000) {$\mathbb{Z}$};
\node[label,color=red,below=of x2] {$x^2$};
\node[group,color=red] (ex1) at (2.000000, 1.000000) {$\mathbb{Z}$};
\draw[->,color=red] (ex1) to (x2);
\node[group,color=red] (x3) at (6.000000, 0.000000) {$\mathbb{Z}$};
\node[label,color=red,below=of x3] {$x^3$};
\node[group,color=red] (ex2) at (4.000000, 1.000000) {$\mathbb{Z}$};
\draw[->,color=red] (ex2) to (x3);
\node[group,color=red] (x4) at (8.000000, 0.000000) {$\mathbb{Z}$};
\node[label,color=red,below=of x4] {$x^4$};
\node[group,color=red] (ex3) at (6.000000, 1.000000) {$\mathbb{Z}$};
\draw[->,color=red] (ex3) to (x4);
\node[group,color=red] (ex4) at (8.000000, 1.000000) {$\mathbb{Z}$};
\end{tikzpicture}
\vspace*{\fill}
\newpage
\vspace*{\fill}
To build the $E_3$ page, we take cohomology with the
    $d_2$ differentials, and we find nothing left but $1$ in the spectral
    sequence.  Hence, $E_3 \cong E_\infty$, and the spectral sequence
    collapses at $E_3$.

    Recall that $E_2^{p, 0} = H^p(\mathbb{C}\mathrm{P}^\infty; H^0(S^1;
    \mathbb{Z})) = H^p(\mathbb{C}\mathrm{P}^\infty; \mathbb{Z})$.  So, we
    can now read off the cohomology of $\mathbb{C}\mathrm{P}^\infty$,
    together with its ring structure: \[H^*(\mathbb{C}\mathrm{P}^\infty;
    \mathbb{Z}) \cong \mathbb{Z}[x],\] where $|x| = 2$.\vspace*{\fill}
\newpage
\vspace*{\fill}
\begin{tikzpicture}[group/.style={},auto]
\draw[use as bounding box,white] (-1.000000, -1.000000) rectangle (9.000000,6.000000);
\path [draw, gray!50, very thin] (0.000000, 0.000000) grid (8.000000, 5.000000);
\draw[->,gray!50,thin] (0, 0) to (9.000000, 0);
\draw[->,gray!50,thin] (0, 0) to (0, 6.000000);
\node[label,gray!50] at (0, -0.5) {$0$};
\node[label,gray!50] at (1, -0.5) {$1$};
\node[label,gray!50] at (2, -0.5) {$2$};
\node[label,gray!50] at (3, -0.5) {$3$};
\node[label,gray!50] at (4, -0.5) {$4$};
\node[label,gray!50] at (5, -0.5) {$5$};
\node[label,gray!50] at (6, -0.5) {$6$};
\node[label,gray!50] at (7, -0.5) {$7$};
\node[label,gray!50] at (8, -0.5) {$8$};
\node[label,gray!50] at (9.000000, -0.5) {$p$};
\node[label,gray!50] at (-0.5, 0) {$0$};
\node[label,gray!50] at (-0.5, 1) {$1$};
\node[label,gray!50] at (-0.5, 2) {$2$};
\node[label,gray!50] at (-0.5, 3) {$3$};
\node[label,gray!50] at (-0.5, 4) {$4$};
\node[label,gray!50] at (-0.5, 5) {$5$};
\node[label,gray!50] at (-0.5, 6.000000) {$q$};
\node[group] (one) at (0.000000, 0.000000) {$\mathbb{Z}$};
\begin{scope}
\clip (-2.000000, 0.000000) rectangle (8.000000, 5.000000);
\end{scope}
\end{tikzpicture}
\vspace*{\fill}
\newpage


\section{$H^* \RP^\infty$ and Gysin sequences}

$K(n)^* B\Z/n$ too?  Then, deducing differentials in the AHSS for $K(n)^* B\Z/n$?

\advancetoleft
\advancetoleft
\section{Unitary groups}
\vspace*{\fill}
Now we will compute the cohomology $H^* BSU$ by
    inductively analyzing related spaces.  We begin by computing the cohomology
    rings $H^* U(n)$, where our primary tool is the fibration \[U(n-1) \to
    U(n) \to \mathbb{C}^{2n} \setminus \{0\} \simeq S^{2n-1}.\]  We
    identify $U(1) \simeq S^1$, which has cohomology $H^* U(1) = \Lambda[e_1]$
    for $|e_1| = 1$.  In general, we claim that $H^* U(n) = \bigotimes_{i \ge
    1} \Lambda[e_{2i-1}]$.  Let's consider the case $n = 3$, for example, whose
    spectral sequence is illustrated at left.
    
    This spectral sequence collapses at this page, using an analysis in two
    parts.  Firstly, consider the indecomposable elements in the fiber column:
    they are all of odd degree, of dimension bounded by $2n-3$.  To support a
    differential, they must cross a large gap to reach the groups in the
    right-hand column, a distance of $2n-1$ across.  This means that
    differentials can occur only on the $E_{2n-1}$-page, of signature $d_{2n-1}:
    E_{2n-1}^{0, q} \to E_{2n-1}^{2n-1, q - 2n}$.  The shift in vertical
    grading forces the differential to land below the $p$-axis, and so it cannot
    exist!

    Secondly, for any decomposable element $\prod_{i \in I} e_i$, we can apply
    the Leibniz rule to get \[d\left( \prod_{i \in I} e_I \right) =
    \sum_{i \in I} \pm d(e_i) \prod_{\substack{j \in I \\ j \ne i}}
    e_j.\]  We just showed that $d(e_i) = 0$ for any $i$, and so the sum
    collapses, determining all those differentials to be zero as well.\vspace*{\fill}
\newpage
\vspace*{\fill}
\begin{tikzpicture}[group/.style={},auto]
\draw[use as bounding box,white] (-1.000000, -1.000000) rectangle (7.000000,11.000000);
\path [draw, gray!50, very thin] (0.000000, 0.000000) grid (6.000000, 10.000000);
\draw[->,gray!50,thin] (0, 0) to (7.000000, 0);
\draw[->,gray!50,thin] (0, 0) to (0, 11.000000);
\node[label,gray!50] at (0, -0.5) {$0$};
\node[label,gray!50] at (1, -0.5) {$1$};
\node[label,gray!50] at (2, -0.5) {$2$};
\node[label,gray!50] at (3, -0.5) {$3$};
\node[label,gray!50] at (4, -0.5) {$4$};
\node[label,gray!50] at (5, -0.5) {$5$};
\node[label,gray!50] at (6, -0.5) {$6$};
\node[label,gray!50] at (7.000000, -0.5) {$p$};
\node[label,gray!50] at (-0.5, 0) {$0$};
\node[label,gray!50] at (-0.5, 1) {$1$};
\node[label,gray!50] at (-0.5, 2) {$2$};
\node[label,gray!50] at (-0.5, 3) {$3$};
\node[label,gray!50] at (-0.5, 4) {$4$};
\node[label,gray!50] at (-0.5, 5) {$5$};
\node[label,gray!50] at (-0.5, 6) {$6$};
\node[label,gray!50] at (-0.5, 7) {$7$};
\node[label,gray!50] at (-0.5, 8) {$8$};
\node[label,gray!50] at (-0.5, 9) {$9$};
\node[label,gray!50] at (-0.5, 10) {$10$};
\node[label,gray!50] at (-0.5, 11.000000) {$q$};
\begin{scope}
\clip (-2.000000, 0.000000) rectangle (6.000000, 10.000000);
\end{scope}
\node[group] (one) at (0.000000, 0.000000) {$\Z$};
\node[group] (e1) at (0.000000, 1.000000) {$\Z$};
\node[label,left=of e1] {$e_1$};
\node[group] (e3) at (0.000000, 3.000000) {$\Z$};
\node[label,left=of e3] {$e_3$};
\node[group] (e1e3) at (0.000000, 4.000000) {$\Z$};
\node[label,left=of e1e3] {$e_1e_3$};
\node[group] (e5) at (5.000000, 0.000000) {$\Z$};
\node[label,below=of e5] {$e_5$};
\node[group] (e1e5) at (5.000000, 1.000000) {$\Z$};
\node[group] (e3e5) at (5.000000, 3.000000) {$\Z$};
\node[group] (e1e3e5) at (5.000000, 4.000000) {$\Z$};
\end{tikzpicture}
\vspace*{\fill}
\newpage
\vspace*{\fill}
Next, we compute the cohomologies $H^* BU(n)$ using
    the fibration $U(n) \to EU(n) \to BU(n)$, where $EU(n) \simeq
    \pt$.  This is very similar to the computation for $\CP^\infty$,
    since the fiber sequence $S^1 \to \C^\infty \setminus \{0\} \to
    \CP^\infty$ is equivalent to $U(1) \to EU(1) \to BU(1)$.  Since the
    total space is contractible, the goal in this game is to clear the board
    by introducing classes in $H^* BU(n)$ to delete the classes already
    present coming from $H^* U(n)$.
    
    At left, we consider the bottom of this spectral sequence for $n \ge 4$. We
    have one chance to delete the class $e_1$, by introducing a class $x_1 \in
    H^* BU(n)$ on page $E_2$, with differential $d(e_1) = x_1$.  Application of
    the Leibniz rule yields a whole host of resulting differentials.\vspace*{\fill}
\newpage
\vspace*{\fill}
\begin{tikzpicture}[group/.style={},auto]
\draw[use as bounding box,white] (-1.000000, -1.000000) rectangle (11.000000,10.000000);
\path [draw, gray!50, very thin] (0.000000, 0.000000) grid (10.000000, 9.000000);
\draw[->,gray!50,thin] (0, 0) to (11.000000, 0);
\draw[->,gray!50,thin] (0, 0) to (0, 10.000000);
\node[label,gray!50] at (0, -0.5) {$0$};
\node[label,gray!50] at (1, -0.5) {$1$};
\node[label,gray!50] at (2, -0.5) {$2$};
\node[label,gray!50] at (3, -0.5) {$3$};
\node[label,gray!50] at (4, -0.5) {$4$};
\node[label,gray!50] at (5, -0.5) {$5$};
\node[label,gray!50] at (6, -0.5) {$6$};
\node[label,gray!50] at (7, -0.5) {$7$};
\node[label,gray!50] at (8, -0.5) {$8$};
\node[label,gray!50] at (9, -0.5) {$9$};
\node[label,gray!50] at (10, -0.5) {$10$};
\node[label,gray!50] at (11.000000, -0.5) {$p$};
\node[label,gray!50] at (-0.5, 0) {$0$};
\node[label,gray!50] at (-0.5, 1) {$1$};
\node[label,gray!50] at (-0.5, 2) {$2$};
\node[label,gray!50] at (-0.5, 3) {$3$};
\node[label,gray!50] at (-0.5, 4) {$4$};
\node[label,gray!50] at (-0.5, 5) {$5$};
\node[label,gray!50] at (-0.5, 6) {$6$};
\node[label,gray!50] at (-0.5, 7) {$7$};
\node[label,gray!50] at (-0.5, 8) {$8$};
\node[label,gray!50] at (-0.5, 9) {$9$};
\node[label,gray!50] at (-0.5, 10.000000) {$q$};
\begin{scope}
\clip (-2.000000, 0.000000) rectangle (10.000000, 9.000000);
\end{scope}
\node[group] (one) at (0.000000, 0.000000) {$\Z$};
\node[group,color=red] (x11) at (2.000000, 0.000000) {$\Z$};
\node[label,color=red,below=of x11] {$x_1$};
\node[group,color=red] (x12) at (4.000000, 0.000000) {$\Z$};
\node[label,color=red,below=of x12] {$x_1^2$};
\node[group,color=red] (x13) at (6.000000, 0.000000) {$\Z$};
\node[label,color=red,below=of x13] {$x_1^3$};
\node[group,color=red] (x14) at (8.000000, 0.000000) {$\Z$};
\node[label,color=red,below=of x14] {$x_1^4$};
\node[group,color=red] (x15) at (10.000000, 0.000000) {$\Z$};
\node[label,color=red,below=of x15] {$x_1^5$};
\node[group] (e9) at (0.000000, 9.000000) {$\Z$};
\node[label,left=of e9] {$e_9$};
\node[group,color=red] (e9x11) at (2.000000, 9.000000) {$\Z$};
\node[group,color=red] (e9x12) at (4.000000, 9.000000) {$\Z$};
\node[group,color=red] (e9x13) at (6.000000, 9.000000) {$\Z$};
\node[group,color=red] (e9x14) at (8.000000, 9.000000) {$\Z$};
\node[group,color=red] (e9x15) at (10.000000, 9.000000) {$\Z$};
\node[group] (e7) at (0.000000, 7.000000) {$\Z$};
\node[label,left=of e7] {$e_7$};
\node[group,color=red] (e7x11) at (2.000000, 7.000000) {$\Z$};
\node[group,color=red] (e7x12) at (4.000000, 7.000000) {$\Z$};
\node[group,color=red] (e7x13) at (6.000000, 7.000000) {$\Z$};
\node[group,color=red] (e7x14) at (8.000000, 7.000000) {$\Z$};
\node[group,color=red] (e7x15) at (10.000000, 7.000000) {$\Z$};
\node[group] (e5) at (0.000000, 5.000000) {$\Z$};
\node[label,left=of e5] {$e_5$};
\node[group,color=red] (e5x11) at (2.000000, 5.000000) {$\Z$};
\node[group,color=red] (e5x12) at (4.000000, 5.000000) {$\Z$};
\node[group,color=red] (e5x13) at (6.000000, 5.000000) {$\Z$};
\node[group,color=red] (e5x14) at (8.000000, 5.000000) {$\Z$};
\node[group,color=red] (e5x15) at (10.000000, 5.000000) {$\Z$};
\node[group] (e3) at (0.000000, 3.000000) {$\Z$};
\node[label,left=of e3] {$e_3$};
\node[group,color=red] (e3x11) at (2.000000, 3.000000) {$\Z$};
\node[group,color=red] (e3x12) at (4.000000, 3.000000) {$\Z$};
\node[group,color=red] (e3x13) at (6.000000, 3.000000) {$\Z$};
\node[group,color=red] (e3x14) at (8.000000, 3.000000) {$\Z$};
\node[group,color=red] (e3x15) at (10.000000, 3.000000) {$\Z$};
\node[group] (e3e7) at (0.000000, 10.000000) {$\Z$};
\node[label,left=of e3e7] {$e_3e_7$};
\node[group,color=red] (e3e7x11) at (2.000000, 10.000000) {$\Z$};
\node[group,color=red] (e3e7x12) at (4.000000, 10.000000) {$\Z$};
\node[group,color=red] (e3e7x13) at (6.000000, 10.000000) {$\Z$};
\node[group,color=red] (e3e7x14) at (8.000000, 10.000000) {$\Z$};
\node[group,color=red] (e3e7x15) at (10.000000, 10.000000) {$\Z$};
\node[group] (e3e5) at (0.000000, 8.000000) {$\Z$};
\node[label,left=of e3e5] {$e_3e_5$};
\node[group,color=red] (e3e5x11) at (2.000000, 8.000000) {$\Z$};
\node[group,color=red] (e3e5x12) at (4.000000, 8.000000) {$\Z$};
\node[group,color=red] (e3e5x13) at (6.000000, 8.000000) {$\Z$};
\node[group,color=red] (e3e5x14) at (8.000000, 8.000000) {$\Z$};
\node[group,color=red] (e3e5x15) at (10.000000, 8.000000) {$\Z$};
\node[group,color=red] (e1) at (0.000000, 1.000000) {$\Z$};
\node[label,color=red,left=of e1] {$e_1$};
\draw[->,color=red] (e1) to (x11);
\node[group,color=red] (e1x11) at (2.000000, 1.000000) {$\Z$};
\draw[->,color=red] (e1x11) to (x12);
\node[group,color=red] (e1x12) at (4.000000, 1.000000) {$\Z$};
\draw[->,color=red] (e1x12) to (x13);
\node[group,color=red] (e1x13) at (6.000000, 1.000000) {$\Z$};
\draw[->,color=red] (e1x13) to (x14);
\node[group,color=red] (e1x14) at (8.000000, 1.000000) {$\Z$};
\draw[->,color=red] (e1x14) to (x15);
\node[group,color=red] (e1x15) at (10.000000, 1.000000) {$\Z$};
\node[group,color=red] (e1e9) at (0.150000, 10.300000) {$\Z$};
\node[label,color=red,left=of e1e9] {$e_1e_9$};
\draw[->,color=red] (e1e9) to (e9x11);
\node[group,color=red] (e1e9x11) at (2.150000, 10.300000) {$\Z$};
\draw[->,color=red] (e1e9x11) to (e9x12);
\node[group,color=red] (e1e9x12) at (4.150000, 10.300000) {$\Z$};
\draw[->,color=red] (e1e9x12) to (e9x13);
\node[group,color=red] (e1e9x13) at (6.150000, 10.300000) {$\Z$};
\draw[->,color=red] (e1e9x13) to (e9x14);
\node[group,color=red] (e1e9x14) at (8.150000, 10.300000) {$\Z$};
\draw[->,color=red] (e1e9x14) to (e9x15);
\node[group,color=red] (e1e9x15) at (10.150000, 10.300000) {$\Z$};
\node[group,color=red] (e1e7) at (0.150000, 8.300000) {$\Z$};
\node[label,color=red,left=of e1e7] {$e_1e_7$};
\draw[->,color=red] (e1e7) to (e7x11);
\node[group,color=red] (e1e7x11) at (2.150000, 8.300000) {$\Z$};
\draw[->,color=red] (e1e7x11) to (e7x12);
\node[group,color=red] (e1e7x12) at (4.150000, 8.300000) {$\Z$};
\draw[->,color=red] (e1e7x12) to (e7x13);
\node[group,color=red] (e1e7x13) at (6.150000, 8.300000) {$\Z$};
\draw[->,color=red] (e1e7x13) to (e7x14);
\node[group,color=red] (e1e7x14) at (8.150000, 8.300000) {$\Z$};
\draw[->,color=red] (e1e7x14) to (e7x15);
\node[group,color=red] (e1e7x15) at (10.150000, 8.300000) {$\Z$};
\node[group,color=red] (e1e5) at (0.000000, 6.000000) {$\Z$};
\node[label,color=red,left=of e1e5] {$e_1e_5$};
\draw[->,color=red] (e1e5) to (e5x11);
\node[group,color=red] (e1e5x11) at (2.000000, 6.000000) {$\Z$};
\draw[->,color=red] (e1e5x11) to (e5x12);
\node[group,color=red] (e1e5x12) at (4.000000, 6.000000) {$\Z$};
\draw[->,color=red] (e1e5x12) to (e5x13);
\node[group,color=red] (e1e5x13) at (6.000000, 6.000000) {$\Z$};
\draw[->,color=red] (e1e5x13) to (e5x14);
\node[group,color=red] (e1e5x14) at (8.000000, 6.000000) {$\Z$};
\draw[->,color=red] (e1e5x14) to (e5x15);
\node[group,color=red] (e1e5x15) at (10.000000, 6.000000) {$\Z$};
\node[group,color=red] (e1e3) at (0.000000, 4.000000) {$\Z$};
\node[label,color=red,left=of e1e3] {$e_1e_3$};
\draw[->,color=red] (e1e3) to (e3x11);
\node[group,color=red] (e1e3x11) at (2.000000, 4.000000) {$\Z$};
\draw[->,color=red] (e1e3x11) to (e3x12);
\node[group,color=red] (e1e3x12) at (4.000000, 4.000000) {$\Z$};
\draw[->,color=red] (e1e3x12) to (e3x13);
\node[group,color=red] (e1e3x13) at (6.000000, 4.000000) {$\Z$};
\draw[->,color=red] (e1e3x13) to (e3x14);
\node[group,color=red] (e1e3x14) at (8.000000, 4.000000) {$\Z$};
\draw[->,color=red] (e1e3x14) to (e3x15);
\node[group,color=red] (e1e3x15) at (10.000000, 4.000000) {$\Z$};
\node[group,color=red] (e1e3e7) at (0.000000, 11.000000) {$\Z$};
\node[label,color=red,left=of e1e3e7] {$e_1e_3e_7$};
\draw[->,color=red] (e1e3e7) to (e3e7x11);
\node[group,color=red] (e1e3e7x11) at (2.000000, 11.000000) {$\Z$};
\draw[->,color=red] (e1e3e7x11) to (e3e7x12);
\node[group,color=red] (e1e3e7x12) at (4.000000, 11.000000) {$\Z$};
\draw[->,color=red] (e1e3e7x12) to (e3e7x13);
\node[group,color=red] (e1e3e7x13) at (6.000000, 11.000000) {$\Z$};
\draw[->,color=red] (e1e3e7x13) to (e3e7x14);
\node[group,color=red] (e1e3e7x14) at (8.000000, 11.000000) {$\Z$};
\draw[->,color=red] (e1e3e7x14) to (e3e7x15);
\node[group,color=red] (e1e3e7x15) at (10.000000, 11.000000) {$\Z$};
\node[group,color=red] (e1e3e5) at (0.150000, 9.300000) {$\Z$};
\node[label,color=red,left=of e1e3e5] {$e_1e_3e_5$};
\draw[->,color=red] (e1e3e5) to (e3e5x11);
\node[group,color=red] (e1e3e5x11) at (2.150000, 9.300000) {$\Z$};
\draw[->,color=red] (e1e3e5x11) to (e3e5x12);
\node[group,color=red] (e1e3e5x12) at (4.150000, 9.300000) {$\Z$};
\draw[->,color=red] (e1e3e5x12) to (e3e5x13);
\node[group,color=red] (e1e3e5x13) at (6.150000, 9.300000) {$\Z$};
\draw[->,color=red] (e1e3e5x13) to (e3e5x14);
\node[group,color=red] (e1e3e5x14) at (8.150000, 9.300000) {$\Z$};
\draw[->,color=red] (e1e3e5x14) to (e3e5x15);
\node[group,color=red] (e1e3e5x15) at (10.150000, 9.300000) {$\Z$};
\end{tikzpicture}
\vspace*{\fill}
\newpage
\vspace*{\fill}
blah.\vspace*{\fill}
\newpage
\vspace*{\fill}
\begin{tikzpicture}[group/.style={},auto]
\draw[use as bounding box,white] (-1.000000, -1.000000) rectangle (11.000000,10.000000);
\path [draw, gray!50, very thin] (0.000000, 0.000000) grid (10.000000, 9.000000);
\draw[->,gray!50,thin] (0, 0) to (11.000000, 0);
\draw[->,gray!50,thin] (0, 0) to (0, 10.000000);
\node[label,gray!50] at (0, -0.5) {$0$};
\node[label,gray!50] at (1, -0.5) {$1$};
\node[label,gray!50] at (2, -0.5) {$2$};
\node[label,gray!50] at (3, -0.5) {$3$};
\node[label,gray!50] at (4, -0.5) {$4$};
\node[label,gray!50] at (5, -0.5) {$5$};
\node[label,gray!50] at (6, -0.5) {$6$};
\node[label,gray!50] at (7, -0.5) {$7$};
\node[label,gray!50] at (8, -0.5) {$8$};
\node[label,gray!50] at (9, -0.5) {$9$};
\node[label,gray!50] at (10, -0.5) {$10$};
\node[label,gray!50] at (11.000000, -0.5) {$p$};
\node[label,gray!50] at (-0.5, 0) {$0$};
\node[label,gray!50] at (-0.5, 1) {$1$};
\node[label,gray!50] at (-0.5, 2) {$2$};
\node[label,gray!50] at (-0.5, 3) {$3$};
\node[label,gray!50] at (-0.5, 4) {$4$};
\node[label,gray!50] at (-0.5, 5) {$5$};
\node[label,gray!50] at (-0.5, 6) {$6$};
\node[label,gray!50] at (-0.5, 7) {$7$};
\node[label,gray!50] at (-0.5, 8) {$8$};
\node[label,gray!50] at (-0.5, 9) {$9$};
\node[label,gray!50] at (-0.5, 10.000000) {$q$};
\node[group] (e3e5) at (0.000000, 8.000000) {$\Z$};
\node[label,left=of e3e5] {$e_3e_5$};
\node[group] (e3) at (0.000000, 3.000000) {$\Z$};
\node[label,left=of e3] {$e_3$};
\node[group] (e5) at (0.000000, 5.000000) {$\Z$};
\node[label,left=of e5] {$e_5$};
\node[group] (e7) at (0.000000, 7.000000) {$\Z$};
\node[label,left=of e7] {$e_7$};
\node[group] (e9) at (0.000000, 9.000000) {$\Z$};
\node[label,left=of e9] {$e_9$};
\node[group] (one) at (0.000000, 0.000000) {$\Z$};
\begin{scope}
\clip (-2.000000, 0.000000) rectangle (10.000000, 9.000000);
\end{scope}
\node[group,color=red] (x21) at (4.150000, 0.300000) {$\Z$};
\node[label,color=red,below=of x21] {$x_2$};
\node[group,color=red] (x22) at (8.150000, 0.300000) {$\Z$};
\node[label,color=red,below=of x22] {$x_2^2$};
\node[group,color=red] (e9x21) at (4.150000, 9.300000) {$\Z$};
\node[group,color=red] (e9x22) at (8.150000, 9.300000) {$\Z$};
\node[group,color=red] (e7x21) at (4.150000, 7.300000) {$\Z$};
\node[group,color=red] (e7x22) at (8.150000, 7.300000) {$\Z$};
\node[group,color=red] (e5x21) at (4.150000, 5.300000) {$\Z$};
\node[group,color=red] (e5x22) at (8.150000, 5.300000) {$\Z$};
\draw[->,color=red] (e3) to (x21);
\node[group,color=red] (e3x21) at (4.150000, 3.300000) {$\Z$};
\draw[->,color=red] (e3x21) to (x22);
\node[group,color=red] (e3x22) at (8.150000, 3.300000) {$\Z$};
\draw[->,color=red] (e3e5) to (e5x21);
\node[group,color=red] (e3e5x21) at (4.150000, 8.300000) {$\Z$};
\draw[->,color=red] (e3e5x21) to (e5x22);
\node[group,color=red] (e3e5x22) at (8.150000, 8.300000) {$\Z$};
\end{tikzpicture}
\vspace*{\fill}
\newpage


$H^* U(n)$, $H^* BU(n)$, $H^* SU(n)$, $H^* BSU(n)$, $H^* BU$, $H^* BSU$

\section{Loopspaces of spheres}

$H^* \Loops S^{2n}$, $H^* \Loops S^{2n+1}$, $H^* \Loops^2 S^{2n+1}$.  Edge homomorphisms.

\section{The Steenrod algebra}

Serre's $H^*(K(\Z/2, q); \F_2)$ and $H^*(K(\Z, q); \F_2)$

\section{$H^*(BU\<6\>; \F_2)$}

Need Kudo transgression.

\section{Unstable homotopy groups of $S^3$}

$\pi_3$, $\pi_4$, $\pi_5 L_{(2)} S^3$
\chapter{Eilenberg-Moore}

Filtration of a bicomplex.  Take a homotopy pullback square $F \to E \to B$ and $F \to X \to B$.  On cohomology, we don't get a pushout; instead, on the level of the derived category of chain complexes, we are taking the derived pushout, giving a spectral sequence from the tensor product of chain complexes to the chain complex of $F$. [[NOTE: How does this need to be graded for multiplicativity?]]

\section{Computing $\Tor$ with Tate resolutions}

In the previous section, it was mentioned that the Eilenberg-Moore spectral sequence is compatible with the multiplicative structure on $\Tor$.  If this is the input to the spectral sequence, then our next question should be: how do we compute this product structure?  Or, even more basically, how do we compute $\Tor$ at all?  In the specific case of $R$ a Noetherian ring and the groups $\Tor^R_{*, *}(R/M, R/N)$, Tate has outlined an extremely useful and simple process for performing this computation, by constructing a DGA whose underlying chain complex is a free resolution of $R/M$.

Let's compute two examples to see Tate's method in action.  First, let's select $R = \Z[x]$, $M = N = \<x\>$, so that we're investigating $\Tor^{\Z[x]}(\Z, \Z)$.  Tate's resolution, like any resolution, begins with the left-hand argument $\Z$, depicted at left as a dot.

At the next stage in the resolution, we introduce a single copy of $R$, which surjects onto $R/M$ by the quotient map $R \onto R/M$.  We haven't deviated from the usual process for building a free resolution yet, but Tate's big idea is that we should be giving these things names as algebra generators as we go.  Since this copy of $R$ lives in degree $0$ of the resolution, and we expect an $R$-algebra in the end, we attach the name ``$1$'' to it, so that its various elements are of the form $r \cdot 1$ for $r \in R$.

To perform the next step, we investigate the kernel of the previous step, depicted beneath the resolution.  The kernel here is the submodule of multiples of $x$, and so we introduce a shifted copy of $R$ in resolution degree $1$, mapping isomorphically into the kernel.  Again, Tate suggests that we give this a name, so we make one up and call it ``$a$''.  The differential connecting degree $1$ to degree $0$ is then described by $da = x$.

At this point, the resolution terminates, since the kernel at filtration degree $1$ is empty.  To compute $\Tor^{\Z[x]}(\Z, \Z)$, we drop the original $\Z$ from the resolution, tensor with $\Z$, and compute the cohomology of the resulting complex.  Since all the differentials hit multiples of $x$, they vanish, and the differential structure evaporates.  The algebra structure, however, does not disappear, and we compute $\Tor^{\Z[x]}(\Z, \Z) = \Z[a] / \<a^2\>$, referred to as an exterior algebra and denoted $\Lambda_{\Z}[x]$.

This example was too short to get interesting, so let's work through another: $\Tor^{\Lambda[x]}(\Z, \Z)$.  

\section{$H^*(\Loops S^{2n+1}; \F_p)$}

$\Loops S^{2n+1}$ using the square $\Loops S^{2n+1} \to \pt \to S^{2n+1}$ and $\Loops S^{2n+1} \to \pt \to S^{2n+1}$.

\section{Complex projective spaces}

Structure of the spectral sequence for $\pt \to \CP^\infty$ pulled back to $S^{2n+1} \to \CP^n$.

\section{The James construction}

The James construction and its filtration, comparison with the particular pullback square $\Loops \Susp X \to \pt \to \Susp X$.

\section{$H^* BU\<6\>$ redux}


\chapter{Co/simplicial objects}

\section{Mayer-Vietoris}

\section{The bar spectral sequence and $K(\F_p; *)$}

Computations of $H_*(K(\F_p, *); \F_p)$ and $K(n)_* K(\F_p, *)$.

Comparison to the Rothenberg-Steenrod spectral sequence (i.e., the Eilenberg-Moore spectral sequence for the square $G \to EG \to BG$, $G \to \pt \to BG$).

\section{The descent spectral sequence}


\chapter{The Adams spectral sequence}

Connection to simplicial objects.

Hill has homework assignments posted at \url{http://people.virginia.edu/~mah7cd/Math885/Homework5.pdf} and \url{http://people.virginia.edu/~mah7cd/Math885/Homework6.pdf}, which can probably be used for more examples.

\section{The dual of the Steenrod algebra}

\section{Resolution by Hopf algebra quotients}

\section{$\pi_* ko^\wedge_2$}

$\pi_* ko^\wedge_{2}$ from $\Ext_{\A(1)}(k, k)$.  This uses $H^*(ko) = A // A(1)$, then a change-of-rings theorem to swap $\Ext_{\A}(\A // \A(1), k)$ for $\Ext_{\A(1)}(k, k)$.  See the tail of \url{http://www.math.ku.dk/~jg/students/masulli.msproject.2011.pdf}, and also Hill's notes at \url{http://people.virginia.edu/~mah7cd/Math885.html} and specifically \url{http://people.virginia.edu/~mah7cd/Math885/Lecture14.pdf}.  The only possible place for a differential is on the guy in $(1, 1)$, called $h_1$, which has the potential to hit a guy in the tower to his left.  This has a cute argument for nonexistence: calling the guy at the bottom of the tower $h_0$, we have $d(h_1) = k h_0^n$ for some $k$ and $n$, so $d(h_0 h_1) = 0 * h_1 + h_0 * k h_0^n$, but $h_0 h_1 = 0$ so $d(h_0 h_1) = 0$ so $k = 0$.  Cuuuute!

\section{$\pi_* ku^\wedge_2$}

This is somehow the same story as $\pi_* ko$, but with an extra layer superimposed?  I very barely remember this...

\section{$ko^* M(2)$}

The ASS over $\A(1)$ for the mod $2$ Moore spectrum is computed at \url{http://people.virginia.edu/~mah7cd/Math885/ExtComps.pdf}; this gives $ko^* M(2)$

\section{Massey products and secondary operations}

\section{Change of rings}

\section{$\pi_* tmf$}

\section{$\pi_{* \le 16} S$}

Hill has lecture notes covering the structure and differentials in the mod 2 Adams spectral sequence through dimension 16 (see Lecture 17).  This is presumably accomplishable by considering just bits of the Steenrod algebra... but I haven't looked.

\chapter{Homotopy fixed points}

Introduction: $H^p(G; \pi_q X) \convergeto \pi_{q-p} X^{hG}$. It comes from filtering $EG_+$ in $X^{hG} = \Hom(EG_+, X)$ using the cellular filtration; the $E_1$-page of the filtration quotients then looks like the cobar complex computing group cohomology for $\pi_q X$ considered as a G-module (so, remembering the $G$-action on the underlying spectrum!).  Alternatively, $X^{hG}$ can be written as a homotopy limit diagram over some category built from $G$, which should give an identical construction after piecing through a construction of 'homotopy limit' using a bar-type construction.  Needs the Adams grading for the multiplicative structure.

\section{Computing $H^*_{gp}(C_n, M)$}

Computing the cohomology of cyclic groups with twisted coefficients is discussed in Weibel 6.2.1-6.2.2; there's a small, periodic resolution that is much better than the cobar construction.

\section{$\pi_* KU^{hC_2}$}

Needs $H^*(C_2; pi_* KU)$, which means knowing $H^*(\RP^\infty; \Z) = (\Z, 0, \Z/2, 0, \Z/2, 0, \ldots)$ in the untwisted case and $H^*(C_2; \Z) = (0, \Z/2, 0, \Z/2, 0, \ldots)$ in the twisted case.  Has a 'multiplication-by-$\eta$' structure that's important for propagating differentials.  The one generating differential is that the guy in degree $(4, 0)$ hits the guy in degree $(3, 3)$ (i.e., hits $\eta^3$), leaving behind the subgroup of $2$-divisible elements.  (How on earth is the existence of this differential proven?)  See Lennart Meier's talk notes, or the photograph I took of Justin's blackboard.

\section{$\pi_* ku^{hC_2}$}

Essentially the same computation, but there's an extra diagonal vanishing line.  This means some elements in negative degrees $-4n$ get missed, and so we don't get $ko$, which has no homotopy in negative degrees.


\section{$\pi_* L_{K(1)} S^0_{(3)}$, $\pi_* L_{K(2)} S^0_{(5)}$}

Hopkins-Miller says $E_n^{hS_n} = L_{K(n)} S^0$.  This computation is accessible for $n = 1$ for sure, but may not involve much of a spectral sequence argument...  It does involve the spectral sequence for composing fixed point functors, but it relies on degeneration.

The $K(2)$-local sphere is ridiculous (Shimomura-Wang, Behrens, ...), but maybe \emph{something} can useful can be said about it without too much hassle.  It may have to get downgraded to a picture.

\chapter{Some pictures of spectral sequences}

Some spectral sequences are too hard to compute with, but now that readers know enough about spectral sequences to interpret diagrams, some completely unproven but important pictures might be appropriate to include in the tail of the book.

\section{$E_2$ for the mod $2$ Adams spectral sequence}

\section{$E_2$ for the $MU$-Adams-Novikov spectral sequence}

\section{Slice spectral sequence and the Kervaire invariant}

\section{The chromatic spectral sequence, stabilizer spectral sequences}

The chromatic spectral sequence, using Wilson's BP sampler.  Existence of the Morava stabilizer spectral sequence, maybe pictures of the stabilizer spectral sequence for the $K(1)$- and $K(2)$-local spheres (at $p \ge 5$?).

\section{The EHP spectral sequence}

Michael Donovan already has a start of a flipbook at \url{http://math.mit.edu/~mdono/\_EHPSS.pdf}, which maybe can be cannibalized for this project.  I pray that he hasn't automated this process and beat me to writing this book!

In fact, much of Haynes Miller's notes on the vector fields on spheres problem deals with and can be restated in terms of the EHP spectral sequence.  Maybe this should be upgraded to its own section, so we can see some unstable phenomena?

\section{Devinatz-Hopkins-Smith and the $X(n)$-Adams spectral sequence}

Is this do-able?  All they show are vanishing lines, but it seems like you should be able to draw a little bit...

\section{$\A(2)$}

Borrow Henriques-Hill?


--- unsorted ---
The May / Bockstein spectral sequences.  This seems hard to produce a good example of, since it mostly organizes computation rather than easing it in any particular way, but maybe it's worth it anyhow.  The organization is pretty cool.
The tangent spectral sequence: formal groups and tangent spaces, $H^*(F; TG) \Rightarrow H^*(F; G)$
Homotopy spectral sequences of spaces (details with $\pi_0$ and $\pi_1$, see Homotopy limits, completions, and localizations by Bousfield and Kan)
Unbased spectral sequences (choose basepoints iteratively, kind of. See Bousfield's Homotopy spectral sequences and obstructions, which does this for unbased cosimplicial spaces.)
Dan Dugger has a paper titled Multiplicative structures on homotopy spectral sequences. Could be worth looking at.
\url{http://neil-strickland.staff.shef.ac.uk/courses/bestiary/ss.pdf}
Pictures of $\mathcal{A}(1)$-resolutions at \url{http://math.wayne.edu/art/}

\end{document}
