\advancetoleft
\section{$H^* \CP^\infty$ redux}
\vspace*{\fill}
Consider the spherical fibration \[S^1 \to \mathbb{C}^\infty \setminus \{0\} \to \mathbb{C}\mathrm{P}^\infty.\]  The total space, $\mathbb{C}^\infty \setminus \{0\} \simeq S^\infty$, is contractible, hence has vanishing cohomology.  The fiber $S^1$ has known cohomology groups, $H^*(S^1; \mathbb{Z}) = \Lambda[e]$.  We know that, $\mathbb{C}\mathbb{P}^\infty$ is connected, and hence we can compute, $H^0(\mathbb{C}\mathbb{P}^\infty; H^* S^1)$ --- it has two free, generators $1$ and $e$ in $q$-degrees $0$ and $1$.\vspace*{\fill}
\newpage
\vspace*{\fill}
\begin{tikzpicture}[group/.style={},auto]
\draw[use as bounding box,white] (-1.000000, -1.000000) rectangle (9.000000,6.000000);
\path [draw, gray!50, very thin] (0.000000, 0.000000) grid (8.000000, 5.000000);
\draw[->,gray!50,thin] (0, 0) to (9.000000, 0);
\draw[->,gray!50,thin] (0, 0) to (0, 6.000000);
\node[label,gray!50] at (0, -0.5) {$0$};
\node[label,gray!50] at (1, -0.5) {$1$};
\node[label,gray!50] at (2, -0.5) {$2$};
\node[label,gray!50] at (3, -0.5) {$3$};
\node[label,gray!50] at (4, -0.5) {$4$};
\node[label,gray!50] at (5, -0.5) {$5$};
\node[label,gray!50] at (6, -0.5) {$6$};
\node[label,gray!50] at (7, -0.5) {$7$};
\node[label,gray!50] at (8, -0.5) {$8$};
\node[label,gray!50] at (9.000000, -0.5) {$p$};
\node[label,gray!50] at (-0.5, 0) {$0$};
\node[label,gray!50] at (-0.5, 1) {$1$};
\node[label,gray!50] at (-0.5, 2) {$2$};
\node[label,gray!50] at (-0.5, 3) {$3$};
\node[label,gray!50] at (-0.5, 4) {$4$};
\node[label,gray!50] at (-0.5, 5) {$5$};
\node[label,gray!50] at (-0.5, 6.000000) {$q$};
\begin{scope}
\clip (-2.000000, 0.000000) rectangle (8.000000, 5.000000);
\end{scope}
\node[group] (one) at (0.000000, 0.000000) {$\mathbb{Z}$};
\node[group,color=red] (e) at (0.000000, 1.000000) {$\mathbb{Z}$};
\node[label,color=red,left=of e] {$e$};
\end{tikzpicture}
\vspace*{\fill}
\newpage
\vspace*{\fill}
The Serre spectral sequence associated to a singular
    theory is a first-quadrant spectral sequence, and hence $E_2^{p, q} = 0$
    whenever $p$ or $q$ is negative.  The differentials have the type signature
    \[d_r: E_r^{p, q} 	o E_r^{p+r, q-r+1}\] and hence if the class $e$ is to be
    killed by a differential --- and it must, since $H^*(S^\infty, \mathbb{Z})
    = \mathbb{Z}$ --- it must happen on this page.  Therefore, there must be a
class $x$ in $E_2^{2, 0} = H^2(\mathbb{C}\mathrm{P}^\infty; H^0(S^1;
\mathbb{Z}))$ with $d_2(e) = x$.\vspace*{\fill}
\newpage
\vspace*{\fill}
\begin{tikzpicture}[group/.style={},auto]
\draw[use as bounding box,white] (-1.000000, -1.000000) rectangle (9.000000,6.000000);
\path [draw, gray!50, very thin] (0.000000, 0.000000) grid (8.000000, 5.000000);
\draw[->,gray!50,thin] (0, 0) to (9.000000, 0);
\draw[->,gray!50,thin] (0, 0) to (0, 6.000000);
\node[label,gray!50] at (0, -0.5) {$0$};
\node[label,gray!50] at (1, -0.5) {$1$};
\node[label,gray!50] at (2, -0.5) {$2$};
\node[label,gray!50] at (3, -0.5) {$3$};
\node[label,gray!50] at (4, -0.5) {$4$};
\node[label,gray!50] at (5, -0.5) {$5$};
\node[label,gray!50] at (6, -0.5) {$6$};
\node[label,gray!50] at (7, -0.5) {$7$};
\node[label,gray!50] at (8, -0.5) {$8$};
\node[label,gray!50] at (9.000000, -0.5) {$p$};
\node[label,gray!50] at (-0.5, 0) {$0$};
\node[label,gray!50] at (-0.5, 1) {$1$};
\node[label,gray!50] at (-0.5, 2) {$2$};
\node[label,gray!50] at (-0.5, 3) {$3$};
\node[label,gray!50] at (-0.5, 4) {$4$};
\node[label,gray!50] at (-0.5, 5) {$5$};
\node[label,gray!50] at (-0.5, 6.000000) {$q$};
\node[group] (e) at (0.000000, 1.000000) {$\mathbb{Z}$};
\node[label,left=of e] {$e$};
\node[group] (one) at (0.000000, 0.000000) {$\mathbb{Z}$};
\begin{scope}
\clip (-2.000000, 0.000000) rectangle (8.000000, 5.000000);
\end{scope}
\node[group,color=red] (x) at (2.000000, 0.000000) {$\mathbb{Z}$};
\node[label,color=red,below=of x] {$x$};
\draw[->,color=red] (e) to (x);
\end{tikzpicture}
\vspace*{\fill}
\newpage
\vspace*{\fill}
But, if $E_2^{2, 0} =
        H^2(\mathbb{C}\mathrm{P}^\infty; H^0(S^1; \mathbb{Z}))$ is nonzero,
        then $E_2^{2, 1} = H^2(\mathbb{C}\mathrm{P}^\infty; H^1(S^1;
        \mathbb{Z}))$ is also nonzero, since $H^0(S^1; \mathbb{Z}) \cong
        H^1(S^1; \mathbb{Z}) \cong \mathbb{Z}$.  The Serre spectral sequence
        is multiplicative, and so we already have a name for this element: $e
        \cdot x$.  Moreover, $d_2$ is a derivation, so \begin{align*} d_2(e
        \cdot x) & = d_2(e) \cdot x + (-1) e \cdot d_2(x) \ & = x^2 + 0 =
        x^2. \end{align*}  For degree reasons, $e \cdot x$ must also be killed
        on the $E_2$-page, and hence $x^2$ must exist in $E_2^{4, 0}$.  This
        pattern continues, as $d_2(e \cdot x^n) = x^{n+1} + (-1) e \cdot n
        x^{n-1} \cdot 0 = x^{n+1}$.\vspace*{\fill}
\newpage
\vspace*{\fill}
\begin{tikzpicture}[group/.style={},auto]
\draw[use as bounding box,white] (-1.000000, -1.000000) rectangle (9.000000,6.000000);
\path [draw, gray!50, very thin] (0.000000, 0.000000) grid (8.000000, 5.000000);
\draw[->,gray!50,thin] (0, 0) to (9.000000, 0);
\draw[->,gray!50,thin] (0, 0) to (0, 6.000000);
\node[label,gray!50] at (0, -0.5) {$0$};
\node[label,gray!50] at (1, -0.5) {$1$};
\node[label,gray!50] at (2, -0.5) {$2$};
\node[label,gray!50] at (3, -0.5) {$3$};
\node[label,gray!50] at (4, -0.5) {$4$};
\node[label,gray!50] at (5, -0.5) {$5$};
\node[label,gray!50] at (6, -0.5) {$6$};
\node[label,gray!50] at (7, -0.5) {$7$};
\node[label,gray!50] at (8, -0.5) {$8$};
\node[label,gray!50] at (9.000000, -0.5) {$p$};
\node[label,gray!50] at (-0.5, 0) {$0$};
\node[label,gray!50] at (-0.5, 1) {$1$};
\node[label,gray!50] at (-0.5, 2) {$2$};
\node[label,gray!50] at (-0.5, 3) {$3$};
\node[label,gray!50] at (-0.5, 4) {$4$};
\node[label,gray!50] at (-0.5, 5) {$5$};
\node[label,gray!50] at (-0.5, 6.000000) {$q$};
\node[group] (x) at (2.000000, 0.000000) {$\mathbb{Z}$};
\node[label,below=of x] {$x$};
\node[group] (e) at (0.000000, 1.000000) {$\mathbb{Z}$};
\node[label,left=of e] {$e$};
\node[group] (one) at (0.000000, 0.000000) {$\mathbb{Z}$};
\begin{scope}
\clip (-2.000000, 0.000000) rectangle (8.000000, 5.000000);
\draw[->] (e) to (x);
\end{scope}
\node[group,color=red] (x2) at (4.000000, 0.000000) {$\mathbb{Z}$};
\node[label,color=red,below=of x2] {$x^2$};
\node[group,color=red] (ex1) at (2.000000, 1.000000) {$\mathbb{Z}$};
\draw[->,color=red] (ex1) to (x2);
\node[group,color=red] (x3) at (6.000000, 0.000000) {$\mathbb{Z}$};
\node[label,color=red,below=of x3] {$x^3$};
\node[group,color=red] (ex2) at (4.000000, 1.000000) {$\mathbb{Z}$};
\draw[->,color=red] (ex2) to (x3);
\node[group,color=red] (x4) at (8.000000, 0.000000) {$\mathbb{Z}$};
\node[label,color=red,below=of x4] {$x^4$};
\node[group,color=red] (ex3) at (6.000000, 1.000000) {$\mathbb{Z}$};
\draw[->,color=red] (ex3) to (x4);
\node[group,color=red] (ex4) at (8.000000, 1.000000) {$\mathbb{Z}$};
\end{tikzpicture}
\vspace*{\fill}
\newpage
\vspace*{\fill}
To build the $E_3$ page, we take cohomology with the
    $d_2$ differentials, and we find nothing left but $1$ in the spectral
    sequence.  Hence, $E_3 \cong E_\infty$, and the spectral sequence
    collapses at $E_3$.

    Recall that $E_2^{p, 0} = H^p(\mathbb{C}\mathrm{P}^\infty; H^0(S^1;
    \mathbb{Z})) = H^p(\mathbb{C}\mathrm{P}^\infty; \mathbb{Z})$.  So, we
    can now read off the cohomology of $\mathbb{C}\mathrm{P}^\infty$,
    together with its ring structure: \[H^*(\mathbb{C}\mathrm{P}^\infty;
    \mathbb{Z}) \cong \mathbb{Z}[x],\] where $|x| = 2$.\vspace*{\fill}
\newpage
\vspace*{\fill}
\begin{tikzpicture}[group/.style={},auto]
\draw[use as bounding box,white] (-1.000000, -1.000000) rectangle (9.000000,6.000000);
\path [draw, gray!50, very thin] (0.000000, 0.000000) grid (8.000000, 5.000000);
\draw[->,gray!50,thin] (0, 0) to (9.000000, 0);
\draw[->,gray!50,thin] (0, 0) to (0, 6.000000);
\node[label,gray!50] at (0, -0.5) {$0$};
\node[label,gray!50] at (1, -0.5) {$1$};
\node[label,gray!50] at (2, -0.5) {$2$};
\node[label,gray!50] at (3, -0.5) {$3$};
\node[label,gray!50] at (4, -0.5) {$4$};
\node[label,gray!50] at (5, -0.5) {$5$};
\node[label,gray!50] at (6, -0.5) {$6$};
\node[label,gray!50] at (7, -0.5) {$7$};
\node[label,gray!50] at (8, -0.5) {$8$};
\node[label,gray!50] at (9.000000, -0.5) {$p$};
\node[label,gray!50] at (-0.5, 0) {$0$};
\node[label,gray!50] at (-0.5, 1) {$1$};
\node[label,gray!50] at (-0.5, 2) {$2$};
\node[label,gray!50] at (-0.5, 3) {$3$};
\node[label,gray!50] at (-0.5, 4) {$4$};
\node[label,gray!50] at (-0.5, 5) {$5$};
\node[label,gray!50] at (-0.5, 6.000000) {$q$};
\node[group] (one) at (0.000000, 0.000000) {$\mathbb{Z}$};
\begin{scope}
\clip (-2.000000, 0.000000) rectangle (8.000000, 5.000000);
\end{scope}
\end{tikzpicture}
\vspace*{\fill}
\newpage
