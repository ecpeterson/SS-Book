\section{Foreword}

Mathematics is a field where computations lead theory, and this is especially evident in the subfield of algebraic topology, which is positively rife with computations.  These often take the form of spectral sequences, which are notorious among students of any field that makes use of homological algebra for being pathologically cryptic and complex.  Nevertheless, their utility is immense, and students, often with much groaning, at least learn to stomach the sight of them, if not fully embrace the idea of computing with one.

There are many reasons spectral sequences are viewed as impossibly complex, large parts of which are due to the following two reasons.  First, spectral sequences are often triply-indexed --- and each index is often infinite, or bi-infinite, or indexed over a group more complicated than the integers!  This means that an enormous amount of information is available in a spectral sequence, which begets the second point: effective computation with a spectral sequence appears to require that one keep an outlandish number of things in mind while working, along with an array of subtle tricks and facts from elsewhere in topology, not presently visible on the page.  In turn, these have lead to a derth of textbooks covering the art of computing with spectral sequences; if they're so difficult to think about, then the situation is even worse when trying to linearize them into writing and then typeset the whole mess.  For this reason, knowing how to compute with spectral sequences is often referred to as an ``oral tradition,'' passed down in ritual form from advisor to student, behind closed doors and with endless scratch paper.

The purpose of this text is to fill this gap.  In conversation with an expert, time plays the role of linearizer, as one watches the spectral sequence play out on a page in real time.  Our goal is to turn these conversations into text, where the linearization instead takes place across pages, in the form of an elementary school student's ``flip book.''  On each page the reader can find a single step of the larger computation highlighted and dissected, then turn to the next to find the diagram slightly modified, as in real-time.  This should dramatically ease the learning curve for students who are interested in spectral sequences but who don't enjoy ready access to lunches with Doug Ravenel and crew.

This book does not have exercises; instead, it is written more like a solutions manual for a text that does not exist.  However, the methods described are extremely general, and the reader looking to try them out for himself should be able to pick a favorite space and plug it into these machines, following roughly the same process to compute its associated invariants.  For this reason, the examples worked here have been selected with illustration kept in mind rather than exhaustiveness.

An important thing to remind the reader of is that spectral sequences, as massive mathematical machines, are designed to take their users' minds off the details of a problem.  Some of these details will be addressed and discussed lightly in the text surrounding the computations, but the uninterested, bored, or befuddled reader should not hesitate to skip over these parts of the text for now.  In the same way, schoolchildren are taught arithmetic algorithms long before they investigate what makes the algorithms tick, and in this intervening period the utility of knowing how to perform long division is not diminished.

We should also immediately mention other textbooks on this subject.  McCleary's book \textit{A User's Guide to Spectral Sequences} is excellent and contains all of the details we omit here and then some.  Mosher \& Tangora's \textit{Cohomology Operations and Applications in Homotopy Theory} centers around the interactions of the Steenrod algebra with spectral sequences, and is rife with the computations that spurred the development of this field.  Every homological algebra textbook in existence (Weibel's \textit{Homological algebra}, Cartan and Eilenberg's \textit{Homological algebra}, \ldots) contains a section on the construction and maintenance of spectral sequences, where technical details can be found.  Hatcher has made available an unfinished book project on spectral sequences at \url{http://www.math.cornell.edu/~hatcher/SSAT/SSATpage.html}.  Miller has published course notes that use in a central way the EHP spectral sequence, available in full at \url{http://www-math.mit.edu/~hrm/papers/} and in the process of being converted to \LaTeX.  Ravenel's \textit{Complex cobordism and stable homotopy groups of spheres} remains the standard reference for the analysis of the beginning of the Adams spectral sequence for the sphere.  And, of course, there are many others.

Finally, this is a draft version of this textbook, compiled on \today.  I'm sure that it's rife with errors, inconsistencies, omissions, and generally confused language, and I would greatly appreciate any or all of corrections, remarks, and expansions.  I can easily be reached at \texttt{ericp@math.berkeley.edu}.  This project progresses slowly, as I tend to work on it only when I'm stuck on and tired of my other mathematical projects, but I hope that it grows into something genuinely useful as it goes.

Drafts of this document are available at \url{http://math.berkeley.edu/\~ericp/ss-book/main.pdf}, and the software used to generate it is available in the directory \url{http://math.berkeley.edu/\~ericp/ss-book/}.

\section{Acknowledgements}

People who directly taught me: Matthew Ando

People who have helped substantially with this book, through contributions or editing: Aaron Mazel-Gee % Akhil Mathews

Topologists whose computations have had a profound influence on me: Mike Hill, Mike Hopkins, Robert Mosher, Justin Noel, Doug Ravenel, Neil Strickland, Martin Tangora, W.\ Steve Wilson

More facilitators: Peter Teichner, Constantin Teleman

Locations and funding sources: UC-Berkeley, MPIM-Bonn
