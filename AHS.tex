\chapter{Atiyah-Hirzebruch-Serre}

\section{The Atiyah-Hirzebruch spectral sequence}

The essential building blocks of the spaces in non-pathological topology (including algebraic topology) are the unit balls $D^n$ of dimension $n$, and their surface spheres $S^{d-1}$ of dimension $(d-1)$.  A topological space $X$ is said to be a CW-complex when it can be decomposed into a sequence of spaces $X^{(d)}$, called $d$-skeleta\footnote{Skeleta is the mathematician's plural of skeleton.}, such that $X^{(-1)} = \pt$ a single point and $X^{(d+1)}$ is formed from $X^{(d)}$ by gluing in (unpointed) $(d+1)$-balls along their $d$-spherical surface shells, and such that $X$ is given as the colimit of the $X^{(d)}$ as $d$ grows large.  These are somehow the most reasonable spaces on which we can ``do homotopy theory,'' and from here on out all our spaces will be assumed to be CW-complexes\footnote{The exact decomposition into the spaces $X^{(d)}$ isn't so important, just that there exists one.}.

This gives an ascending filtration of $X$ by $X^{(d)}$ which is Hausdorff (the condition on $X^{(-1)}$) and exhaustive (the condition $X = \colim_d X^{(d)}$).  Moreover, the filtration quotients are easy to compute: the cofiber of the map $X^{(d-1)} \into X^{(d)}$ collapses the $(d-1)$-skeleton to a point, to which all the $d$-cells get attached, resulting in a bouquet of spheres $X^{(d)} / X^{(d-1)} \simeq \bigvee_\alpha S^d_\alpha$ in filtration grading $d$.  Selecting our favorite homology theory $h_*$ and cohomology theory $h^*$, this gives a pair of spectral sequences with signatures
\begin{align*}
E^1_{s, t} = h_s \left(\bigvee_\alpha S^t_\alpha\right) = h_{s-t}(\pt) & \convergeto h_s X, & d^r: E^1_{s, t} & \to E^1_{s-1, t-r}, \\
E_1^{s, t} = h^s \left( \bigvee_\alpha S^t_\alpha \right) = h^{s-t}(\pt) & \convergeto h^s X, & d_r: E_1^{s, t} & \to E_1^{s+1, t+r}.
\end{align*}
In fact, we can do better: the differential on the first pages of these spectral sequences is exactly the differential that appears in the $(s-t)$th degree of the cellular chain complex for computing cohomology with $h^{s-t}(\pt)$-coefficients.  This spectral sequence also carries the structure of a $h^*(\pt)$-module in the case that $h$ takes its values in rings, though not with this grading.  \TODO: Straighten out this grading discussion.  Putting all this together produces the more familiar form of these spectral sequences:
\begin{align*}
E^2_{p, q} = H^{cell}_p(X; h_q(\pt)) & \convergeto H_{p+q} X, & d^r_{p, q}: E^r_{p, q} & \to E^r_{p-r, q+r-1}, \\
E_2^{p, q} = H_{cell}^p(X; h^q(\pt)) & \convergeto H^{p+q} X, & d_r^{p, q}: E_r^{p, q} & \to E_r^{p+r, q-r+1}.
\end{align*}

\section{$H^* \CP^\infty$}

The motivic cell decomposition.

\section{$H^* \RP^\infty$}

The motivic cell decomposition.

\section{$KU^* B\Z/2$}

Even-concentrated, but has extension problems.  See Strickland's bestiary.  This might be hard to do before the Gysin sequence description of $h^* \RP^\infty$...

\section{The Serre spectral sequence}

The $E_1$-page is easy, but $d_1$ is hard.  Multiplicative structure.

\advancetoleft
\section{$H^* \CP^\infty$ redux}
\vspace*{\fill}
Consider the spherical fibration \[S^1 \to \mathbb{C}^\infty \setminus \{0\} \to \mathbb{C}\mathrm{P}^\infty.\]  The total space, $\mathbb{C}^\infty \setminus \{0\} \simeq S^\infty$, is contractible, hence has vanishing cohomology.  The fiber $S^1$ has known cohomology groups, $H^*(S^1; \mathbb{Z}) = \Lambda[e]$.  We know that, $\mathbb{C}\mathbb{P}^\infty$ is connected, and hence we can compute, $H^0(\mathbb{C}\mathbb{P}^\infty; H^* S^1)$ --- it has two free, generators $1$ and $e$ in $q$-degrees $0$ and $1$.\vspace*{\fill}
\newpage
\vspace*{\fill}
\begin{tikzpicture}[group/.style={},auto]
\draw[use as bounding box,white] (-1.000000, -1.000000) rectangle (9.000000,6.000000);
\path [draw, gray!50, very thin] (0.000000, 0.000000) grid (8.000000, 5.000000);
\draw[->,gray!50,thin] (0, 0) to (9.000000, 0);
\draw[->,gray!50,thin] (0, 0) to (0, 6.000000);
\node[label,gray!50] at (0, -0.5) {$0$};
\node[label,gray!50] at (1, -0.5) {$1$};
\node[label,gray!50] at (2, -0.5) {$2$};
\node[label,gray!50] at (3, -0.5) {$3$};
\node[label,gray!50] at (4, -0.5) {$4$};
\node[label,gray!50] at (5, -0.5) {$5$};
\node[label,gray!50] at (6, -0.5) {$6$};
\node[label,gray!50] at (7, -0.5) {$7$};
\node[label,gray!50] at (8, -0.5) {$8$};
\node[label,gray!50] at (9.000000, -0.5) {$p$};
\node[label,gray!50] at (-0.5, 0) {$0$};
\node[label,gray!50] at (-0.5, 1) {$1$};
\node[label,gray!50] at (-0.5, 2) {$2$};
\node[label,gray!50] at (-0.5, 3) {$3$};
\node[label,gray!50] at (-0.5, 4) {$4$};
\node[label,gray!50] at (-0.5, 5) {$5$};
\node[label,gray!50] at (-0.5, 6.000000) {$q$};
\begin{scope}
\clip (-2.000000, 0.000000) rectangle (8.000000, 5.000000);
\end{scope}
\node[group] (one) at (0.000000, 0.000000) {$\mathbb{Z}$};
\node[group,color=red] (e) at (0.000000, 1.000000) {$\mathbb{Z}$};
\node[label,color=red,left=of e] {$e$};
\end{tikzpicture}
\vspace*{\fill}
\newpage
\vspace*{\fill}
The Serre spectral sequence associated to a singular
    theory is a first-quadrant spectral sequence, and hence $E_2^{p, q} = 0$
    whenever $p$ or $q$ is negative.  The differentials have the type signature
    \[d_r: E_r^{p, q} 	o E_r^{p+r, q-r+1}\] and hence if the class $e$ is to be
    killed by a differential --- and it must, since $H^*(S^\infty, \mathbb{Z})
    = \mathbb{Z}$ --- it must happen on this page.  Therefore, there must be a
class $x$ in $E_2^{2, 0} = H^2(\mathbb{C}\mathrm{P}^\infty; H^0(S^1;
\mathbb{Z}))$ with $d_2(e) = x$.\vspace*{\fill}
\newpage
\vspace*{\fill}
\begin{tikzpicture}[group/.style={},auto]
\draw[use as bounding box,white] (-1.000000, -1.000000) rectangle (9.000000,6.000000);
\path [draw, gray!50, very thin] (0.000000, 0.000000) grid (8.000000, 5.000000);
\draw[->,gray!50,thin] (0, 0) to (9.000000, 0);
\draw[->,gray!50,thin] (0, 0) to (0, 6.000000);
\node[label,gray!50] at (0, -0.5) {$0$};
\node[label,gray!50] at (1, -0.5) {$1$};
\node[label,gray!50] at (2, -0.5) {$2$};
\node[label,gray!50] at (3, -0.5) {$3$};
\node[label,gray!50] at (4, -0.5) {$4$};
\node[label,gray!50] at (5, -0.5) {$5$};
\node[label,gray!50] at (6, -0.5) {$6$};
\node[label,gray!50] at (7, -0.5) {$7$};
\node[label,gray!50] at (8, -0.5) {$8$};
\node[label,gray!50] at (9.000000, -0.5) {$p$};
\node[label,gray!50] at (-0.5, 0) {$0$};
\node[label,gray!50] at (-0.5, 1) {$1$};
\node[label,gray!50] at (-0.5, 2) {$2$};
\node[label,gray!50] at (-0.5, 3) {$3$};
\node[label,gray!50] at (-0.5, 4) {$4$};
\node[label,gray!50] at (-0.5, 5) {$5$};
\node[label,gray!50] at (-0.5, 6.000000) {$q$};
\node[group] (e) at (0.000000, 1.000000) {$\mathbb{Z}$};
\node[label,left=of e] {$e$};
\node[group] (one) at (0.000000, 0.000000) {$\mathbb{Z}$};
\begin{scope}
\clip (-2.000000, 0.000000) rectangle (8.000000, 5.000000);
\end{scope}
\node[group,color=red] (x) at (2.000000, 0.000000) {$\mathbb{Z}$};
\node[label,color=red,below=of x] {$x$};
\draw[->,color=red] (e) to (x);
\end{tikzpicture}
\vspace*{\fill}
\newpage
\vspace*{\fill}
But, if $E_2^{2, 0} =
        H^2(\mathbb{C}\mathrm{P}^\infty; H^0(S^1; \mathbb{Z}))$ is nonzero,
        then $E_2^{2, 1} = H^2(\mathbb{C}\mathrm{P}^\infty; H^1(S^1;
        \mathbb{Z}))$ is also nonzero, since $H^0(S^1; \mathbb{Z}) \cong
        H^1(S^1; \mathbb{Z}) \cong \mathbb{Z}$.  The Serre spectral sequence
        is multiplicative, and so we already have a name for this element: $e
        \cdot x$.  Moreover, $d_2$ is a derivation, so \begin{align*} d_2(e
        \cdot x) & = d_2(e) \cdot x + (-1) e \cdot d_2(x) \ & = x^2 + 0 =
        x^2. \end{align*}  For degree reasons, $e \cdot x$ must also be killed
        on the $E_2$-page, and hence $x^2$ must exist in $E_2^{4, 0}$.  This
        pattern continues, as $d_2(e \cdot x^n) = x^{n+1} + (-1) e \cdot n
        x^{n-1} \cdot 0 = x^{n+1}$.\vspace*{\fill}
\newpage
\vspace*{\fill}
\begin{tikzpicture}[group/.style={},auto]
\draw[use as bounding box,white] (-1.000000, -1.000000) rectangle (9.000000,6.000000);
\path [draw, gray!50, very thin] (0.000000, 0.000000) grid (8.000000, 5.000000);
\draw[->,gray!50,thin] (0, 0) to (9.000000, 0);
\draw[->,gray!50,thin] (0, 0) to (0, 6.000000);
\node[label,gray!50] at (0, -0.5) {$0$};
\node[label,gray!50] at (1, -0.5) {$1$};
\node[label,gray!50] at (2, -0.5) {$2$};
\node[label,gray!50] at (3, -0.5) {$3$};
\node[label,gray!50] at (4, -0.5) {$4$};
\node[label,gray!50] at (5, -0.5) {$5$};
\node[label,gray!50] at (6, -0.5) {$6$};
\node[label,gray!50] at (7, -0.5) {$7$};
\node[label,gray!50] at (8, -0.5) {$8$};
\node[label,gray!50] at (9.000000, -0.5) {$p$};
\node[label,gray!50] at (-0.5, 0) {$0$};
\node[label,gray!50] at (-0.5, 1) {$1$};
\node[label,gray!50] at (-0.5, 2) {$2$};
\node[label,gray!50] at (-0.5, 3) {$3$};
\node[label,gray!50] at (-0.5, 4) {$4$};
\node[label,gray!50] at (-0.5, 5) {$5$};
\node[label,gray!50] at (-0.5, 6.000000) {$q$};
\node[group] (x) at (2.000000, 0.000000) {$\mathbb{Z}$};
\node[label,below=of x] {$x$};
\node[group] (e) at (0.000000, 1.000000) {$\mathbb{Z}$};
\node[label,left=of e] {$e$};
\node[group] (one) at (0.000000, 0.000000) {$\mathbb{Z}$};
\begin{scope}
\clip (-2.000000, 0.000000) rectangle (8.000000, 5.000000);
\draw[->] (e) to (x);
\end{scope}
\node[group,color=red] (x2) at (4.000000, 0.000000) {$\mathbb{Z}$};
\node[label,color=red,below=of x2] {$x^2$};
\node[group,color=red] (ex1) at (2.000000, 1.000000) {$\mathbb{Z}$};
\draw[->,color=red] (ex1) to (x2);
\node[group,color=red] (x3) at (6.000000, 0.000000) {$\mathbb{Z}$};
\node[label,color=red,below=of x3] {$x^3$};
\node[group,color=red] (ex2) at (4.000000, 1.000000) {$\mathbb{Z}$};
\draw[->,color=red] (ex2) to (x3);
\node[group,color=red] (x4) at (8.000000, 0.000000) {$\mathbb{Z}$};
\node[label,color=red,below=of x4] {$x^4$};
\node[group,color=red] (ex3) at (6.000000, 1.000000) {$\mathbb{Z}$};
\draw[->,color=red] (ex3) to (x4);
\node[group,color=red] (ex4) at (8.000000, 1.000000) {$\mathbb{Z}$};
\end{tikzpicture}
\vspace*{\fill}
\newpage
\vspace*{\fill}
To build the $E_3$ page, we take cohomology with the
    $d_2$ differentials, and we find nothing left but $1$ in the spectral
    sequence.  Hence, $E_3 \cong E_\infty$, and the spectral sequence
    collapses at $E_3$.

    Recall that $E_2^{p, 0} = H^p(\mathbb{C}\mathrm{P}^\infty; H^0(S^1;
    \mathbb{Z})) = H^p(\mathbb{C}\mathrm{P}^\infty; \mathbb{Z})$.  So, we
    can now read off the cohomology of $\mathbb{C}\mathrm{P}^\infty$,
    together with its ring structure: \[H^*(\mathbb{C}\mathrm{P}^\infty;
    \mathbb{Z}) \cong \mathbb{Z}[x],\] where $|x| = 2$.\vspace*{\fill}
\newpage
\vspace*{\fill}
\begin{tikzpicture}[group/.style={},auto]
\draw[use as bounding box,white] (-1.000000, -1.000000) rectangle (9.000000,6.000000);
\path [draw, gray!50, very thin] (0.000000, 0.000000) grid (8.000000, 5.000000);
\draw[->,gray!50,thin] (0, 0) to (9.000000, 0);
\draw[->,gray!50,thin] (0, 0) to (0, 6.000000);
\node[label,gray!50] at (0, -0.5) {$0$};
\node[label,gray!50] at (1, -0.5) {$1$};
\node[label,gray!50] at (2, -0.5) {$2$};
\node[label,gray!50] at (3, -0.5) {$3$};
\node[label,gray!50] at (4, -0.5) {$4$};
\node[label,gray!50] at (5, -0.5) {$5$};
\node[label,gray!50] at (6, -0.5) {$6$};
\node[label,gray!50] at (7, -0.5) {$7$};
\node[label,gray!50] at (8, -0.5) {$8$};
\node[label,gray!50] at (9.000000, -0.5) {$p$};
\node[label,gray!50] at (-0.5, 0) {$0$};
\node[label,gray!50] at (-0.5, 1) {$1$};
\node[label,gray!50] at (-0.5, 2) {$2$};
\node[label,gray!50] at (-0.5, 3) {$3$};
\node[label,gray!50] at (-0.5, 4) {$4$};
\node[label,gray!50] at (-0.5, 5) {$5$};
\node[label,gray!50] at (-0.5, 6.000000) {$q$};
\node[group] (one) at (0.000000, 0.000000) {$\mathbb{Z}$};
\begin{scope}
\clip (-2.000000, 0.000000) rectangle (8.000000, 5.000000);
\end{scope}
\end{tikzpicture}
\vspace*{\fill}
\newpage


\section{$H^* \RP^\infty$ and Gysin sequences}

$K(n)^* B\Z/n$ too?  Then, deducing differentials in the AHSS for $K(n)^* B\Z/n$?

\advancetoleft
\advancetoleft
\section{Unitary groups}
\vspace*{\fill}
Now we will compute the cohomology $H^* BSU$ by
    inductively analyzing related spaces.  We begin by computing the cohomology
    rings $H^* U(n)$, where our primary tool is the fibration \[U(n-1) \to
    U(n) \to \mathbb{C}^{2n} \setminus \{0\} \simeq S^{2n-1}.\]  We
    identify $U(1) \simeq S^1$, which has cohomology $H^* U(1) = \Lambda[e_1]$
    for $|e_1| = 1$.  In general, we claim that $H^* U(n) = \bigotimes_{i \ge
    1} \Lambda[e_{2i-1}]$.  Let's consider the case $n = 3$, for example, whose
    spectral sequence is illustrated at left.
    
    This spectral sequence collapses at this page, using an analysis in two
    parts.  Firstly, consider the indecomposable elements in the fiber column:
    they are all of odd degree, of dimension bounded by $2n-3$.  To support a
    differential, they must cross a large gap to reach the groups in the
    right-hand column, a distance of $2n-1$ across.  This means that
    differentials can occur only on the $E_{2n-1}$-page, of signature $d_{2n-1}:
    E_{2n-1}^{0, q} \to E_{2n-1}^{2n-1, q - 2n}$.  The shift in vertical
    grading forces the differential to land below the $p$-axis, and so it cannot
    exist!

    Secondly, for any decomposable element $\prod_{i \in I} e_i$, we can apply
    the Leibniz rule to get \[d\left( \prod_{i \in I} e_I \right) =
    \sum_{i \in I} \pm d(e_i) \prod_{\substack{j \in I \\ j \ne i}}
    e_j.\]  We just showed that $d(e_i) = 0$ for any $i$, and so the sum
    collapses, determining all those differentials to be zero as well.\vspace*{\fill}
\newpage
\vspace*{\fill}
\begin{tikzpicture}[group/.style={},auto]
\draw[use as bounding box,white] (-1.000000, -1.000000) rectangle (7.000000,11.000000);
\path [draw, gray!50, very thin] (0.000000, 0.000000) grid (6.000000, 10.000000);
\draw[->,gray!50,thin] (0, 0) to (7.000000, 0);
\draw[->,gray!50,thin] (0, 0) to (0, 11.000000);
\node[label,gray!50] at (0, -0.5) {$0$};
\node[label,gray!50] at (1, -0.5) {$1$};
\node[label,gray!50] at (2, -0.5) {$2$};
\node[label,gray!50] at (3, -0.5) {$3$};
\node[label,gray!50] at (4, -0.5) {$4$};
\node[label,gray!50] at (5, -0.5) {$5$};
\node[label,gray!50] at (6, -0.5) {$6$};
\node[label,gray!50] at (7.000000, -0.5) {$p$};
\node[label,gray!50] at (-0.5, 0) {$0$};
\node[label,gray!50] at (-0.5, 1) {$1$};
\node[label,gray!50] at (-0.5, 2) {$2$};
\node[label,gray!50] at (-0.5, 3) {$3$};
\node[label,gray!50] at (-0.5, 4) {$4$};
\node[label,gray!50] at (-0.5, 5) {$5$};
\node[label,gray!50] at (-0.5, 6) {$6$};
\node[label,gray!50] at (-0.5, 7) {$7$};
\node[label,gray!50] at (-0.5, 8) {$8$};
\node[label,gray!50] at (-0.5, 9) {$9$};
\node[label,gray!50] at (-0.5, 10) {$10$};
\node[label,gray!50] at (-0.5, 11.000000) {$q$};
\begin{scope}
\clip (-2.000000, 0.000000) rectangle (6.000000, 10.000000);
\end{scope}
\node[group] (one) at (0.000000, 0.000000) {$\Z$};
\node[group] (e1) at (0.000000, 1.000000) {$\Z$};
\node[label,left=of e1] {$e_1$};
\node[group] (e3) at (0.000000, 3.000000) {$\Z$};
\node[label,left=of e3] {$e_3$};
\node[group] (e1e3) at (0.000000, 4.000000) {$\Z$};
\node[label,left=of e1e3] {$e_1e_3$};
\node[group] (e5) at (5.000000, 0.000000) {$\Z$};
\node[label,below=of e5] {$e_5$};
\node[group] (e1e5) at (5.000000, 1.000000) {$\Z$};
\node[group] (e3e5) at (5.000000, 3.000000) {$\Z$};
\node[group] (e1e3e5) at (5.000000, 4.000000) {$\Z$};
\end{tikzpicture}
\vspace*{\fill}
\newpage
\vspace*{\fill}
Next, we compute the cohomologies $H^* BU(n)$ using
    the fibration $U(n) \to EU(n) \to BU(n)$, where $EU(n) \simeq
    \pt$.  This is very similar to the computation for $\CP^\infty$,
    since the fiber sequence $S^1 \to \C^\infty \setminus \{0\} \to
    \CP^\infty$ is equivalent to $U(1) \to EU(1) \to BU(1)$.  Since the
    total space is contractible, the goal in this game is to clear the board
    by introducing classes in $H^* BU(n)$ to delete the classes already
    present coming from $H^* U(n)$.
    
    At left, we consider the bottom of this spectral sequence for $n \ge 4$. We
    have one chance to delete the class $e_1$, by introducing a class $x_1 \in
    H^* BU(n)$ on page $E_2$, with differential $d(e_1) = x_1$.  Application of
    the Leibniz rule yields a whole host of resulting differentials.\vspace*{\fill}
\newpage
\vspace*{\fill}
\begin{tikzpicture}[group/.style={},auto]
\draw[use as bounding box,white] (-1.000000, -1.000000) rectangle (11.000000,10.000000);
\path [draw, gray!50, very thin] (0.000000, 0.000000) grid (10.000000, 9.000000);
\draw[->,gray!50,thin] (0, 0) to (11.000000, 0);
\draw[->,gray!50,thin] (0, 0) to (0, 10.000000);
\node[label,gray!50] at (0, -0.5) {$0$};
\node[label,gray!50] at (1, -0.5) {$1$};
\node[label,gray!50] at (2, -0.5) {$2$};
\node[label,gray!50] at (3, -0.5) {$3$};
\node[label,gray!50] at (4, -0.5) {$4$};
\node[label,gray!50] at (5, -0.5) {$5$};
\node[label,gray!50] at (6, -0.5) {$6$};
\node[label,gray!50] at (7, -0.5) {$7$};
\node[label,gray!50] at (8, -0.5) {$8$};
\node[label,gray!50] at (9, -0.5) {$9$};
\node[label,gray!50] at (10, -0.5) {$10$};
\node[label,gray!50] at (11.000000, -0.5) {$p$};
\node[label,gray!50] at (-0.5, 0) {$0$};
\node[label,gray!50] at (-0.5, 1) {$1$};
\node[label,gray!50] at (-0.5, 2) {$2$};
\node[label,gray!50] at (-0.5, 3) {$3$};
\node[label,gray!50] at (-0.5, 4) {$4$};
\node[label,gray!50] at (-0.5, 5) {$5$};
\node[label,gray!50] at (-0.5, 6) {$6$};
\node[label,gray!50] at (-0.5, 7) {$7$};
\node[label,gray!50] at (-0.5, 8) {$8$};
\node[label,gray!50] at (-0.5, 9) {$9$};
\node[label,gray!50] at (-0.5, 10.000000) {$q$};
\begin{scope}
\clip (-2.000000, 0.000000) rectangle (10.000000, 9.000000);
\end{scope}
\node[group] (one) at (0.000000, 0.000000) {$\Z$};
\node[group,color=red] (x11) at (2.000000, 0.000000) {$\Z$};
\node[label,color=red,below=of x11] {$x_1$};
\node[group,color=red] (x12) at (4.000000, 0.000000) {$\Z$};
\node[label,color=red,below=of x12] {$x_1^2$};
\node[group,color=red] (x13) at (6.000000, 0.000000) {$\Z$};
\node[label,color=red,below=of x13] {$x_1^3$};
\node[group,color=red] (x14) at (8.000000, 0.000000) {$\Z$};
\node[label,color=red,below=of x14] {$x_1^4$};
\node[group,color=red] (x15) at (10.000000, 0.000000) {$\Z$};
\node[label,color=red,below=of x15] {$x_1^5$};
\node[group] (e9) at (0.000000, 9.000000) {$\Z$};
\node[label,left=of e9] {$e_9$};
\node[group,color=red] (e9x11) at (2.000000, 9.000000) {$\Z$};
\node[group,color=red] (e9x12) at (4.000000, 9.000000) {$\Z$};
\node[group,color=red] (e9x13) at (6.000000, 9.000000) {$\Z$};
\node[group,color=red] (e9x14) at (8.000000, 9.000000) {$\Z$};
\node[group,color=red] (e9x15) at (10.000000, 9.000000) {$\Z$};
\node[group] (e7) at (0.000000, 7.000000) {$\Z$};
\node[label,left=of e7] {$e_7$};
\node[group,color=red] (e7x11) at (2.000000, 7.000000) {$\Z$};
\node[group,color=red] (e7x12) at (4.000000, 7.000000) {$\Z$};
\node[group,color=red] (e7x13) at (6.000000, 7.000000) {$\Z$};
\node[group,color=red] (e7x14) at (8.000000, 7.000000) {$\Z$};
\node[group,color=red] (e7x15) at (10.000000, 7.000000) {$\Z$};
\node[group] (e5) at (0.000000, 5.000000) {$\Z$};
\node[label,left=of e5] {$e_5$};
\node[group,color=red] (e5x11) at (2.000000, 5.000000) {$\Z$};
\node[group,color=red] (e5x12) at (4.000000, 5.000000) {$\Z$};
\node[group,color=red] (e5x13) at (6.000000, 5.000000) {$\Z$};
\node[group,color=red] (e5x14) at (8.000000, 5.000000) {$\Z$};
\node[group,color=red] (e5x15) at (10.000000, 5.000000) {$\Z$};
\node[group] (e3) at (0.000000, 3.000000) {$\Z$};
\node[label,left=of e3] {$e_3$};
\node[group,color=red] (e3x11) at (2.000000, 3.000000) {$\Z$};
\node[group,color=red] (e3x12) at (4.000000, 3.000000) {$\Z$};
\node[group,color=red] (e3x13) at (6.000000, 3.000000) {$\Z$};
\node[group,color=red] (e3x14) at (8.000000, 3.000000) {$\Z$};
\node[group,color=red] (e3x15) at (10.000000, 3.000000) {$\Z$};
\node[group] (e3e7) at (0.000000, 10.000000) {$\Z$};
\node[label,left=of e3e7] {$e_3e_7$};
\node[group,color=red] (e3e7x11) at (2.000000, 10.000000) {$\Z$};
\node[group,color=red] (e3e7x12) at (4.000000, 10.000000) {$\Z$};
\node[group,color=red] (e3e7x13) at (6.000000, 10.000000) {$\Z$};
\node[group,color=red] (e3e7x14) at (8.000000, 10.000000) {$\Z$};
\node[group,color=red] (e3e7x15) at (10.000000, 10.000000) {$\Z$};
\node[group] (e3e5) at (0.000000, 8.000000) {$\Z$};
\node[label,left=of e3e5] {$e_3e_5$};
\node[group,color=red] (e3e5x11) at (2.000000, 8.000000) {$\Z$};
\node[group,color=red] (e3e5x12) at (4.000000, 8.000000) {$\Z$};
\node[group,color=red] (e3e5x13) at (6.000000, 8.000000) {$\Z$};
\node[group,color=red] (e3e5x14) at (8.000000, 8.000000) {$\Z$};
\node[group,color=red] (e3e5x15) at (10.000000, 8.000000) {$\Z$};
\node[group,color=red] (e1) at (0.000000, 1.000000) {$\Z$};
\node[label,color=red,left=of e1] {$e_1$};
\draw[->,color=red] (e1) to (x11);
\node[group,color=red] (e1x11) at (2.000000, 1.000000) {$\Z$};
\draw[->,color=red] (e1x11) to (x12);
\node[group,color=red] (e1x12) at (4.000000, 1.000000) {$\Z$};
\draw[->,color=red] (e1x12) to (x13);
\node[group,color=red] (e1x13) at (6.000000, 1.000000) {$\Z$};
\draw[->,color=red] (e1x13) to (x14);
\node[group,color=red] (e1x14) at (8.000000, 1.000000) {$\Z$};
\draw[->,color=red] (e1x14) to (x15);
\node[group,color=red] (e1x15) at (10.000000, 1.000000) {$\Z$};
\node[group,color=red] (e1e9) at (0.150000, 10.300000) {$\Z$};
\node[label,color=red,left=of e1e9] {$e_1e_9$};
\draw[->,color=red] (e1e9) to (e9x11);
\node[group,color=red] (e1e9x11) at (2.150000, 10.300000) {$\Z$};
\draw[->,color=red] (e1e9x11) to (e9x12);
\node[group,color=red] (e1e9x12) at (4.150000, 10.300000) {$\Z$};
\draw[->,color=red] (e1e9x12) to (e9x13);
\node[group,color=red] (e1e9x13) at (6.150000, 10.300000) {$\Z$};
\draw[->,color=red] (e1e9x13) to (e9x14);
\node[group,color=red] (e1e9x14) at (8.150000, 10.300000) {$\Z$};
\draw[->,color=red] (e1e9x14) to (e9x15);
\node[group,color=red] (e1e9x15) at (10.150000, 10.300000) {$\Z$};
\node[group,color=red] (e1e7) at (0.150000, 8.300000) {$\Z$};
\node[label,color=red,left=of e1e7] {$e_1e_7$};
\draw[->,color=red] (e1e7) to (e7x11);
\node[group,color=red] (e1e7x11) at (2.150000, 8.300000) {$\Z$};
\draw[->,color=red] (e1e7x11) to (e7x12);
\node[group,color=red] (e1e7x12) at (4.150000, 8.300000) {$\Z$};
\draw[->,color=red] (e1e7x12) to (e7x13);
\node[group,color=red] (e1e7x13) at (6.150000, 8.300000) {$\Z$};
\draw[->,color=red] (e1e7x13) to (e7x14);
\node[group,color=red] (e1e7x14) at (8.150000, 8.300000) {$\Z$};
\draw[->,color=red] (e1e7x14) to (e7x15);
\node[group,color=red] (e1e7x15) at (10.150000, 8.300000) {$\Z$};
\node[group,color=red] (e1e5) at (0.000000, 6.000000) {$\Z$};
\node[label,color=red,left=of e1e5] {$e_1e_5$};
\draw[->,color=red] (e1e5) to (e5x11);
\node[group,color=red] (e1e5x11) at (2.000000, 6.000000) {$\Z$};
\draw[->,color=red] (e1e5x11) to (e5x12);
\node[group,color=red] (e1e5x12) at (4.000000, 6.000000) {$\Z$};
\draw[->,color=red] (e1e5x12) to (e5x13);
\node[group,color=red] (e1e5x13) at (6.000000, 6.000000) {$\Z$};
\draw[->,color=red] (e1e5x13) to (e5x14);
\node[group,color=red] (e1e5x14) at (8.000000, 6.000000) {$\Z$};
\draw[->,color=red] (e1e5x14) to (e5x15);
\node[group,color=red] (e1e5x15) at (10.000000, 6.000000) {$\Z$};
\node[group,color=red] (e1e3) at (0.000000, 4.000000) {$\Z$};
\node[label,color=red,left=of e1e3] {$e_1e_3$};
\draw[->,color=red] (e1e3) to (e3x11);
\node[group,color=red] (e1e3x11) at (2.000000, 4.000000) {$\Z$};
\draw[->,color=red] (e1e3x11) to (e3x12);
\node[group,color=red] (e1e3x12) at (4.000000, 4.000000) {$\Z$};
\draw[->,color=red] (e1e3x12) to (e3x13);
\node[group,color=red] (e1e3x13) at (6.000000, 4.000000) {$\Z$};
\draw[->,color=red] (e1e3x13) to (e3x14);
\node[group,color=red] (e1e3x14) at (8.000000, 4.000000) {$\Z$};
\draw[->,color=red] (e1e3x14) to (e3x15);
\node[group,color=red] (e1e3x15) at (10.000000, 4.000000) {$\Z$};
\node[group,color=red] (e1e3e7) at (0.000000, 11.000000) {$\Z$};
\node[label,color=red,left=of e1e3e7] {$e_1e_3e_7$};
\draw[->,color=red] (e1e3e7) to (e3e7x11);
\node[group,color=red] (e1e3e7x11) at (2.000000, 11.000000) {$\Z$};
\draw[->,color=red] (e1e3e7x11) to (e3e7x12);
\node[group,color=red] (e1e3e7x12) at (4.000000, 11.000000) {$\Z$};
\draw[->,color=red] (e1e3e7x12) to (e3e7x13);
\node[group,color=red] (e1e3e7x13) at (6.000000, 11.000000) {$\Z$};
\draw[->,color=red] (e1e3e7x13) to (e3e7x14);
\node[group,color=red] (e1e3e7x14) at (8.000000, 11.000000) {$\Z$};
\draw[->,color=red] (e1e3e7x14) to (e3e7x15);
\node[group,color=red] (e1e3e7x15) at (10.000000, 11.000000) {$\Z$};
\node[group,color=red] (e1e3e5) at (0.150000, 9.300000) {$\Z$};
\node[label,color=red,left=of e1e3e5] {$e_1e_3e_5$};
\draw[->,color=red] (e1e3e5) to (e3e5x11);
\node[group,color=red] (e1e3e5x11) at (2.150000, 9.300000) {$\Z$};
\draw[->,color=red] (e1e3e5x11) to (e3e5x12);
\node[group,color=red] (e1e3e5x12) at (4.150000, 9.300000) {$\Z$};
\draw[->,color=red] (e1e3e5x12) to (e3e5x13);
\node[group,color=red] (e1e3e5x13) at (6.150000, 9.300000) {$\Z$};
\draw[->,color=red] (e1e3e5x13) to (e3e5x14);
\node[group,color=red] (e1e3e5x14) at (8.150000, 9.300000) {$\Z$};
\draw[->,color=red] (e1e3e5x14) to (e3e5x15);
\node[group,color=red] (e1e3e5x15) at (10.150000, 9.300000) {$\Z$};
\end{tikzpicture}
\vspace*{\fill}
\newpage
\vspace*{\fill}
blah.\vspace*{\fill}
\newpage
\vspace*{\fill}
\begin{tikzpicture}[group/.style={},auto]
\draw[use as bounding box,white] (-1.000000, -1.000000) rectangle (11.000000,10.000000);
\path [draw, gray!50, very thin] (0.000000, 0.000000) grid (10.000000, 9.000000);
\draw[->,gray!50,thin] (0, 0) to (11.000000, 0);
\draw[->,gray!50,thin] (0, 0) to (0, 10.000000);
\node[label,gray!50] at (0, -0.5) {$0$};
\node[label,gray!50] at (1, -0.5) {$1$};
\node[label,gray!50] at (2, -0.5) {$2$};
\node[label,gray!50] at (3, -0.5) {$3$};
\node[label,gray!50] at (4, -0.5) {$4$};
\node[label,gray!50] at (5, -0.5) {$5$};
\node[label,gray!50] at (6, -0.5) {$6$};
\node[label,gray!50] at (7, -0.5) {$7$};
\node[label,gray!50] at (8, -0.5) {$8$};
\node[label,gray!50] at (9, -0.5) {$9$};
\node[label,gray!50] at (10, -0.5) {$10$};
\node[label,gray!50] at (11.000000, -0.5) {$p$};
\node[label,gray!50] at (-0.5, 0) {$0$};
\node[label,gray!50] at (-0.5, 1) {$1$};
\node[label,gray!50] at (-0.5, 2) {$2$};
\node[label,gray!50] at (-0.5, 3) {$3$};
\node[label,gray!50] at (-0.5, 4) {$4$};
\node[label,gray!50] at (-0.5, 5) {$5$};
\node[label,gray!50] at (-0.5, 6) {$6$};
\node[label,gray!50] at (-0.5, 7) {$7$};
\node[label,gray!50] at (-0.5, 8) {$8$};
\node[label,gray!50] at (-0.5, 9) {$9$};
\node[label,gray!50] at (-0.5, 10.000000) {$q$};
\node[group] (e3e5) at (0.000000, 8.000000) {$\Z$};
\node[label,left=of e3e5] {$e_3e_5$};
\node[group] (e3) at (0.000000, 3.000000) {$\Z$};
\node[label,left=of e3] {$e_3$};
\node[group] (e5) at (0.000000, 5.000000) {$\Z$};
\node[label,left=of e5] {$e_5$};
\node[group] (e7) at (0.000000, 7.000000) {$\Z$};
\node[label,left=of e7] {$e_7$};
\node[group] (e9) at (0.000000, 9.000000) {$\Z$};
\node[label,left=of e9] {$e_9$};
\node[group] (one) at (0.000000, 0.000000) {$\Z$};
\begin{scope}
\clip (-2.000000, 0.000000) rectangle (10.000000, 9.000000);
\end{scope}
\node[group,color=red] (x21) at (4.150000, 0.300000) {$\Z$};
\node[label,color=red,below=of x21] {$x_2$};
\node[group,color=red] (x22) at (8.150000, 0.300000) {$\Z$};
\node[label,color=red,below=of x22] {$x_2^2$};
\node[group,color=red] (e9x21) at (4.150000, 9.300000) {$\Z$};
\node[group,color=red] (e9x22) at (8.150000, 9.300000) {$\Z$};
\node[group,color=red] (e7x21) at (4.150000, 7.300000) {$\Z$};
\node[group,color=red] (e7x22) at (8.150000, 7.300000) {$\Z$};
\node[group,color=red] (e5x21) at (4.150000, 5.300000) {$\Z$};
\node[group,color=red] (e5x22) at (8.150000, 5.300000) {$\Z$};
\draw[->,color=red] (e3) to (x21);
\node[group,color=red] (e3x21) at (4.150000, 3.300000) {$\Z$};
\draw[->,color=red] (e3x21) to (x22);
\node[group,color=red] (e3x22) at (8.150000, 3.300000) {$\Z$};
\draw[->,color=red] (e3e5) to (e5x21);
\node[group,color=red] (e3e5x21) at (4.150000, 8.300000) {$\Z$};
\draw[->,color=red] (e3e5x21) to (e5x22);
\node[group,color=red] (e3e5x22) at (8.150000, 8.300000) {$\Z$};
\end{tikzpicture}
\vspace*{\fill}
\newpage


$H^* U(n)$, $H^* BU(n)$, $H^* SU(n)$, $H^* BSU(n)$, $H^* BU$, $H^* BSU$

\section{Loopspaces of spheres}

$H^* \Loops S^{2n}$, $H^* \Loops S^{2n+1}$, $H^* \Loops^2 S^{2n+1}$.  Edge homomorphisms.

\section{The Steenrod algebra}

Serre's $H^*(K(\Z/2, q); \F_2)$ and $H^*(K(\Z, q); \F_2)$

\section{$H^*(BU\<6\>; \F_2)$}

Need Kudo transgression.

\section{Unstable homotopy groups of $S^3$}

$\pi_3$, $\pi_4$, $\pi_5 L_{(2)} S^3$